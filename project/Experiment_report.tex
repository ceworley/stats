\documentclass[]{article}
\usepackage{lmodern}
\usepackage{amssymb,amsmath}
\usepackage{ifxetex,ifluatex}
\usepackage{fixltx2e} % provides \textsubscript
\ifnum 0\ifxetex 1\fi\ifluatex 1\fi=0 % if pdftex
  \usepackage[T1]{fontenc}
  \usepackage[utf8]{inputenc}
\else % if luatex or xelatex
  \ifxetex
    \usepackage{mathspec}
  \else
    \usepackage{fontspec}
  \fi
  \defaultfontfeatures{Ligatures=TeX,Scale=MatchLowercase}
\fi
% use upquote if available, for straight quotes in verbatim environments
\IfFileExists{upquote.sty}{\usepackage{upquote}}{}
% use microtype if available
\IfFileExists{microtype.sty}{%
\usepackage{microtype}
\UseMicrotypeSet[protrusion]{basicmath} % disable protrusion for tt fonts
}{}
\usepackage[margin=1in]{geometry}
\usepackage{hyperref}
\hypersetup{unicode=true,
            pdftitle={Estimations of Distance to Moon Anchored by Radius of Earth and Distance to Sun},
            pdfborder={0 0 0},
            breaklinks=true}
\urlstyle{same}  % don't use monospace font for urls
\usepackage{color}
\usepackage{fancyvrb}
\newcommand{\VerbBar}{|}
\newcommand{\VERB}{\Verb[commandchars=\\\{\}]}
\DefineVerbatimEnvironment{Highlighting}{Verbatim}{commandchars=\\\{\}}
% Add ',fontsize=\small' for more characters per line
\usepackage{framed}
\definecolor{shadecolor}{RGB}{248,248,248}
\newenvironment{Shaded}{\begin{snugshade}}{\end{snugshade}}
\newcommand{\AlertTok}[1]{\textcolor[rgb]{0.94,0.16,0.16}{#1}}
\newcommand{\AnnotationTok}[1]{\textcolor[rgb]{0.56,0.35,0.01}{\textbf{\textit{#1}}}}
\newcommand{\AttributeTok}[1]{\textcolor[rgb]{0.77,0.63,0.00}{#1}}
\newcommand{\BaseNTok}[1]{\textcolor[rgb]{0.00,0.00,0.81}{#1}}
\newcommand{\BuiltInTok}[1]{#1}
\newcommand{\CharTok}[1]{\textcolor[rgb]{0.31,0.60,0.02}{#1}}
\newcommand{\CommentTok}[1]{\textcolor[rgb]{0.56,0.35,0.01}{\textit{#1}}}
\newcommand{\CommentVarTok}[1]{\textcolor[rgb]{0.56,0.35,0.01}{\textbf{\textit{#1}}}}
\newcommand{\ConstantTok}[1]{\textcolor[rgb]{0.00,0.00,0.00}{#1}}
\newcommand{\ControlFlowTok}[1]{\textcolor[rgb]{0.13,0.29,0.53}{\textbf{#1}}}
\newcommand{\DataTypeTok}[1]{\textcolor[rgb]{0.13,0.29,0.53}{#1}}
\newcommand{\DecValTok}[1]{\textcolor[rgb]{0.00,0.00,0.81}{#1}}
\newcommand{\DocumentationTok}[1]{\textcolor[rgb]{0.56,0.35,0.01}{\textbf{\textit{#1}}}}
\newcommand{\ErrorTok}[1]{\textcolor[rgb]{0.64,0.00,0.00}{\textbf{#1}}}
\newcommand{\ExtensionTok}[1]{#1}
\newcommand{\FloatTok}[1]{\textcolor[rgb]{0.00,0.00,0.81}{#1}}
\newcommand{\FunctionTok}[1]{\textcolor[rgb]{0.00,0.00,0.00}{#1}}
\newcommand{\ImportTok}[1]{#1}
\newcommand{\InformationTok}[1]{\textcolor[rgb]{0.56,0.35,0.01}{\textbf{\textit{#1}}}}
\newcommand{\KeywordTok}[1]{\textcolor[rgb]{0.13,0.29,0.53}{\textbf{#1}}}
\newcommand{\NormalTok}[1]{#1}
\newcommand{\OperatorTok}[1]{\textcolor[rgb]{0.81,0.36,0.00}{\textbf{#1}}}
\newcommand{\OtherTok}[1]{\textcolor[rgb]{0.56,0.35,0.01}{#1}}
\newcommand{\PreprocessorTok}[1]{\textcolor[rgb]{0.56,0.35,0.01}{\textit{#1}}}
\newcommand{\RegionMarkerTok}[1]{#1}
\newcommand{\SpecialCharTok}[1]{\textcolor[rgb]{0.00,0.00,0.00}{#1}}
\newcommand{\SpecialStringTok}[1]{\textcolor[rgb]{0.31,0.60,0.02}{#1}}
\newcommand{\StringTok}[1]{\textcolor[rgb]{0.31,0.60,0.02}{#1}}
\newcommand{\VariableTok}[1]{\textcolor[rgb]{0.00,0.00,0.00}{#1}}
\newcommand{\VerbatimStringTok}[1]{\textcolor[rgb]{0.31,0.60,0.02}{#1}}
\newcommand{\WarningTok}[1]{\textcolor[rgb]{0.56,0.35,0.01}{\textbf{\textit{#1}}}}
\usepackage{longtable,booktabs}
\usepackage{graphicx,grffile}
\makeatletter
\def\maxwidth{\ifdim\Gin@nat@width>\linewidth\linewidth\else\Gin@nat@width\fi}
\def\maxheight{\ifdim\Gin@nat@height>\textheight\textheight\else\Gin@nat@height\fi}
\makeatother
% Scale images if necessary, so that they will not overflow the page
% margins by default, and it is still possible to overwrite the defaults
% using explicit options in \includegraphics[width, height, ...]{}
\setkeys{Gin}{width=\maxwidth,height=\maxheight,keepaspectratio}
\IfFileExists{parskip.sty}{%
\usepackage{parskip}
}{% else
\setlength{\parindent}{0pt}
\setlength{\parskip}{6pt plus 2pt minus 1pt}
}
\setlength{\emergencystretch}{3em}  % prevent overfull lines
\providecommand{\tightlist}{%
  \setlength{\itemsep}{0pt}\setlength{\parskip}{0pt}}
\setcounter{secnumdepth}{0}
% Redefines (sub)paragraphs to behave more like sections
\ifx\paragraph\undefined\else
\let\oldparagraph\paragraph
\renewcommand{\paragraph}[1]{\oldparagraph{#1}\mbox{}}
\fi
\ifx\subparagraph\undefined\else
\let\oldsubparagraph\subparagraph
\renewcommand{\subparagraph}[1]{\oldsubparagraph{#1}\mbox{}}
\fi

%%% Use protect on footnotes to avoid problems with footnotes in titles
\let\rmarkdownfootnote\footnote%
\def\footnote{\protect\rmarkdownfootnote}

%%% Change title format to be more compact
\usepackage{titling}

% Create subtitle command for use in maketitle
\providecommand{\subtitle}[1]{
  \posttitle{
    \begin{center}\large#1\end{center}
    }
}

\setlength{\droptitle}{-2em}

  \title{Estimations of Distance to Moon Anchored by Radius of Earth and Distance
to Sun}
    \pretitle{\vspace{\droptitle}\centering\huge}
  \posttitle{\par}
    \author{}
    \preauthor{}\postauthor{}
    \date{}
    \predate{}\postdate{}
  

\begin{document}
\maketitle

\hypertarget{abstract}{%
\subsection{Abstract}\label{abstract}}

Anchoring is demonstrated in a Bunker Hill class of 24 students. Each
student was asked to estimate the distance to the moon; group 1 was
anchored on the distance to the sun (9e7 miles) while group 2 anchored
with radius of Earth (3e3 miles). Due to heavy skew in group 1, the
normality-based pvalue was 0.1 while permutation-based methods allowed
us to conclude that anchoring caused an effect
(\(p\text{-value} = 0.03\)).

\hypertarget{background}{%
\subsection{Background}\label{background}}

Say some things about anchoring.

Statistical tests often assume sampling distributions are normal. This
condition is easily met with moderate sample sizes from nearly any
population (even with a population as unnormal as bimodal or exponential
distributions). However, with small sample sizes, the normal
approximation to sampling distributions easily fails.

Luckily, permutation tests are more rubust. With a modern version of R,
permutation tests are intuitive to code. The 24 students were tasked
with determining the results of a permutation test on their own original
data; 10 were able to implement the permutation test compared to 15 who
were able to implement a two-proportion \(t\)-test.

\begin{Shaded}
\begin{Highlighting}[]
\KeywordTok{prop.test}\NormalTok{(}\DataTypeTok{x=}\KeywordTok{c}\NormalTok{(}\DecValTok{10}\NormalTok{,}\DecValTok{15}\NormalTok{),}\DataTypeTok{n=}\KeywordTok{c}\NormalTok{(}\DecValTok{24}\NormalTok{,}\DecValTok{24}\NormalTok{))}
\end{Highlighting}
\end{Shaded}

\begin{verbatim}
## 
##  2-sample test for equality of proportions with continuity
##  correction
## 
## data:  c(10, 15) out of c(24, 24)
## X-squared = 1.3357, df = 1, p-value = 0.2478
## alternative hypothesis: two.sided
## 95 percent confidence interval:
##  -0.5264379  0.1097712
## sample estimates:
##    prop 1    prop 2 
## 0.4166667 0.6250000
\end{verbatim}

It seems (from the \(p\)-value) more data is necessary to determine a
preference. It should be mentioned that the class these students were in
focused on normal-based techniques (like standard introductory classes
usually do). However, the concept of permutation (randomization) tests
was introduced in the first chapter with a card-shuffling
description/embodiment of the procedure. (make reference to OpenIntro)

\hypertarget{experimental-methods}{%
\subsection{Experimental Methods}\label{experimental-methods}}

Every participant was asked to estimate the distance to the moon. By
having the instructions on cards, and asking the responses to be written
on the cards, all participants could participate simultaneously. The
participants were given approximately 30 seconds to write an answer. All
participants responded.

The two types of cards gave different hints. The groups (Earth and Sun)
were randomly assigned by shuffling the cards. By using cards, it was
easy to guarantee equal sample sizes. Half of the participants were told
the radius of Earth (\(3,000\) miles) while the other half were told the
distance to the Sun (\(90,000,000\) miles). The cards were made with the
printout that can be generated (or modified) with the R script below.

\begin{Shaded}
\begin{Highlighting}[]
\CommentTok{#This code will create a page that can be printed (saved in the current working directory). }
\CommentTok{#The resulting pdf/printout is in the supplemental materials.}
\KeywordTok{pdf}\NormalTok{(}\StringTok{"Moon_cards.pdf"}\NormalTok{,}\DataTypeTok{width=}\FloatTok{8.5}\NormalTok{,}\DataTypeTok{height=}\DecValTok{11}\NormalTok{,}\DataTypeTok{paper=}\StringTok{"letter"}\NormalTok{,}\DataTypeTok{title =} \StringTok{"Moon cards printout"}\NormalTok{) }
\KeywordTok{par}\NormalTok{(}\DataTypeTok{mar =} \KeywordTok{c}\NormalTok{(}\DecValTok{0}\NormalTok{,}\DecValTok{0}\NormalTok{,}\DecValTok{0}\NormalTok{,}\DecValTok{0}\NormalTok{))}
\KeywordTok{plot.new}\NormalTok{()}
\NormalTok{fig.align=}\StringTok{"center"}
\ControlFlowTok{for}\NormalTok{(xpos }\ControlFlowTok{in} \KeywordTok{seq}\NormalTok{(}\DecValTok{0}\NormalTok{,}\DecValTok{1}\NormalTok{,}\DecValTok{1}\OperatorTok{/}\DecValTok{3}\NormalTok{))\{}
  \KeywordTok{abline}\NormalTok{(}\DataTypeTok{v =}\NormalTok{ xpos)}
\NormalTok{\}}
\ControlFlowTok{for}\NormalTok{(ypos }\ControlFlowTok{in} \KeywordTok{seq}\NormalTok{(}\DecValTok{0}\NormalTok{,}\DecValTok{1}\NormalTok{,}\DecValTok{1}\OperatorTok{/}\DecValTok{4}\NormalTok{))\{}
  \KeywordTok{abline}\NormalTok{(}\DataTypeTok{h =}\NormalTok{ ypos)}
\NormalTok{\}}
\ControlFlowTok{for}\NormalTok{(x }\ControlFlowTok{in} \KeywordTok{seq}\NormalTok{(}\DecValTok{0}\NormalTok{,}\DecValTok{2}\OperatorTok{/}\DecValTok{3}\NormalTok{,}\DecValTok{1}\OperatorTok{/}\DecValTok{3}\NormalTok{))\{}
  \ControlFlowTok{for}\NormalTok{(y }\ControlFlowTok{in} \KeywordTok{seq}\NormalTok{(}\DecValTok{0}\NormalTok{,}\DecValTok{3}\OperatorTok{/}\DecValTok{4}\NormalTok{,}\DecValTok{1}\OperatorTok{/}\DecValTok{4}\NormalTok{))\{}
    \KeywordTok{text}\NormalTok{(x}\FloatTok{+0.16}\NormalTok{,y}\FloatTok{+0.20}\NormalTok{,}\StringTok{"Please estimate the distance}\CharTok{\textbackslash{}n}\StringTok{to the moon }\CharTok{\textbackslash{}n}\StringTok{in miles."}\NormalTok{, }\DataTypeTok{cex=}\FloatTok{0.8}\NormalTok{)}
  \KeywordTok{lines}\NormalTok{(}\KeywordTok{c}\NormalTok{(x}\FloatTok{+0.05}\NormalTok{,x}\FloatTok{+0.3333-0.05}\NormalTok{),}\KeywordTok{c}\NormalTok{(y}\FloatTok{+0.07}\NormalTok{,y}\FloatTok{+0.07}\NormalTok{))}
  \ControlFlowTok{if}\NormalTok{ (y }\OperatorTok\StringTok{ }\KeywordTok{c}\NormalTok{(}\DecValTok{0}\NormalTok{,}\FloatTok{0.5}\NormalTok{))\{}
    \KeywordTok{text}\NormalTok{(x}\FloatTok{+0.16}\NormalTok{,y}\FloatTok{+0.03}\NormalTok{,}\StringTok{"Hint:}\CharTok{\textbackslash{}n}\StringTok{The sun is 90,000,000 miles away."}\NormalTok{, }\DataTypeTok{cex=}\FloatTok{0.8}\NormalTok{)}
\NormalTok{  \} }\ControlFlowTok{else}\NormalTok{ \{}
    \KeywordTok{text}\NormalTok{(x}\FloatTok{+0.16}\NormalTok{,y}\FloatTok{+0.03}\NormalTok{,}\StringTok{"Hint:}\CharTok{\textbackslash{}n}\StringTok{The radius of Earth is 3,000 miles."}\NormalTok{, }\DataTypeTok{cex=}\FloatTok{0.8}\NormalTok{)}
\NormalTok{  \}}
\NormalTok{  \}}
\NormalTok{\}}
\KeywordTok{dev.off}\NormalTok{()}
\end{Highlighting}
\end{Shaded}

\includegraphics{Experiment_report_files/figure-latex/card2-1.pdf}

The statistical analysis took two forms: Welch's \(t\) test and an
approximate permutation method.

\hypertarget{results}{%
\subsection{Results}\label{results}}

The sample data are shown here.

\begin{Shaded}
\begin{Highlighting}[]
\NormalTok{earth =}\StringTok{ }\KeywordTok{c}\NormalTok{(}\DecValTok{20000}\NormalTok{,}\DecValTok{500000}\NormalTok{,}\DecValTok{30000}\NormalTok{,}\DecValTok{10000}\NormalTok{,}\DecValTok{122500}\NormalTok{,}\DecValTok{100000}\NormalTok{,}\FloatTok{2e6}\NormalTok{,}\DecValTok{7800}\NormalTok{,}\DecValTok{6500}\NormalTok{,}\DecValTok{15000}\NormalTok{,}\DecValTok{120000}\NormalTok{,}\DecValTok{400000}\NormalTok{)}
\NormalTok{sun =}\StringTok{ }\KeywordTok{c}\NormalTok{(}\FloatTok{4e7}\NormalTok{,}\FloatTok{2e6}\NormalTok{,}\FloatTok{5e5}\NormalTok{,}\FloatTok{6e7}\NormalTok{,}\FloatTok{4e6}\NormalTok{,}\FloatTok{5e3}\NormalTok{,}\FloatTok{3e5}\NormalTok{,}\FloatTok{1e5}\NormalTok{,}\FloatTok{6e6}\NormalTok{,}\FloatTok{3e6}\NormalTok{,}\FloatTok{1e7}\NormalTok{,}\FloatTok{1e4}\NormalTok{)}
\KeywordTok{hist}\NormalTok{(earth)}
\end{Highlighting}
\end{Shaded}

\includegraphics{Experiment_report_files/figure-latex/sample_data-1.pdf}

\begin{Shaded}
\begin{Highlighting}[]
\KeywordTok{hist}\NormalTok{(sun)}
\end{Highlighting}
\end{Shaded}

\includegraphics{Experiment_report_files/figure-latex/sample_data-2.pdf}
Notice both of these groups appear to have a heavy skew, so we should be
wary of using an approach that assumes (repeated) small samples would be
normally distributed. We continue with the Welch's \(t\) test as an
exercise.

We find the means and sample standard deviations.

\begin{Shaded}
\begin{Highlighting}[]
\NormalTok{xbar1 =}\StringTok{ }\KeywordTok{mean}\NormalTok{(earth)}
\NormalTok{s1 =}\StringTok{ }\KeywordTok{sd}\NormalTok{(earth)}
\NormalTok{xbar2 =}\StringTok{ }\KeywordTok{mean}\NormalTok{(sun)}
\NormalTok{s2 =}\StringTok{ }\KeywordTok{sd}\NormalTok{(sun)}
\end{Highlighting}
\end{Shaded}

\begin{longtable}[]{@{}ccc@{}}
\toprule
Group & Mean & Standard deviation\tabularnewline
\midrule
\endhead
Earth & 2.7765\times 10\^{}\{5\} &
5.6617216\times 10\^{}\{5\}\tabularnewline
\bottomrule
\end{longtable}

\hypertarget{discussion}{%
\subsection{Discussion}\label{discussion}}


\end{document}
