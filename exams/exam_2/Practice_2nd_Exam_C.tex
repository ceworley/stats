\documentclass[12pt,letterpaper,addpoints]{exam}
\usepackage[utf8]{inputenc}
\usepackage{amsmath}
\usepackage{amsfonts}
\usepackage{amssymb}
\usepackage{amsthm}
\usepackage{graphicx}
\usepackage{tabularx}
\usepackage[left=2cm,right=2cm,top=2cm,bottom=2cm]{geometry}
\usepackage{multicol}
\usepackage{multirow,array}
\usepackage{newtxtext,newtxmath}
\usepackage{lastpage}
\usepackage{enumitem}
\usepackage{pdfpages}
\newcolumntype{Y}{>{\centering\arraybackslash}X}
\firstpageheader{}{}{\includegraphics[scale=0.5]{BHCClogoBW.jpg}\hspace{-40pt}\vspace{-50pt}}
\firstpagefooter{}{}{Page \thepage ~of \pageref{LastPage}}
\runningheader{ \textsc{Math 181 2nd Exam Practice C}}{}{ \textsc{Spring 2019}}%{\includegraphics[scale=.5]{BHCClogoBW.jpg}\vspace{-10pt}}%\hspace{-60pt}\vspace{-10pt}}
\runningheadrule
\runningfooter{}{}{Page \thepage ~of \pageref{LastPage}}
\renewcommand{\thequestion}{{\bf Q\arabic{question}}}
%\renewcommand{\questionlabel}{{\thequestion .}}
%\pointformat{\fbox{\themarginpoints \,pt}}
%\pointsinrightmargin
%\setlength{\rightpointsmargin}{1.5cm}
%\pointsinmargin
%\setlength{\marginpointssep}{10pt}

\begin{document}

\newcommand{\AND}{~\textsc{and}~}
\newcommand{\OR}{~\textsc{or}~}

\begin{center}
\text{ }\\
\vspace{40pt}
\textsc{{\Huge Math 181 2nd Exam Practice C}}\\\vspace{5pt}
\textsc{{\large Spring 2019}}\\
\vspace{60pt}
\makebox[\textwidth]{\large Name:\enspace\hrulefill}\\
\vspace{40pt}
\fbox{\fbox{\begin{minipage}{6in}
\vspace{0.2in}
\begin{itemize}
\item Write your {\bf full name} on the line above.
\item Show your work. Incorrect answers with work can receive partial credit.
\item Attempt every question; showing you understand the question earns some credit.
\item If you run out of room for an answer, continue on the back of the page. Before doing so, write ``see back'' with a circle around it.
\item You can use 1 page (front and back) of notes.
\item You can use (and probably need) a calculator.
\item You can use the Geogebra Scientific Calculator instead of a calculator. You need to put your phone on {\bf airplane mode} and then within the application, start {\bf exam mode}; you should see a green bar with a timer counting up.
\item If a question is confusing or ambiguous, please ask for clarification; however, you will not be told how to  answer the question.
\item {\bf Box your final answer}.
\item A formula sheet is attached to this test.
\vspace{0.2in}
\end{itemize}
\end{minipage}
\begin{minipage}{0.2in}\text{ }\end{minipage}}}
\vfill
Do not write in this grade table.\\
\gradetable[h][questions]\\
\vfill
\end{center}

%%%%%%%%%%%%%%%%%%%%%%%%%%%%%%%%%%%%%%%%%%%%%%%%%%%%%%%%%%%%%%%%%%%%%%%%%%%%%%%
\newcommand{\N}[2]{\mathcal{N}\big(#1\,,~#2\big)}
\newcommand{\Bern}[1]{\texttt{Bern}\big(#1\big)}
\newcommand{\Geo}[1]{\texttt{Geo}\big(#1\big)}
\newcommand{\B}[2]{\mathcal{B}\big(#1\,,~#2\big)}
\newcommand{\Sampp}[2]{\texttt{Samp}_{\hat{p}}\big(#1,~#2\big)}
%\newcommand{\Sampx}[2]{\texttt{Samp}_{\hat{p}}\big(#1,~#2\big)}

\newpage

\begin{multicols}{2}

%\rhead{\textsc{Definitions and Formulas}}

{\bf Normal Distribution:}\\
$X \sim \N{\mu}{\sigma}$\\
$\mu = \text{ population mean} $\\
$\sigma = \text{ population standard deviation} $\\
$x = \text{ possible value of }X $\\
$\ell = \text{ percentile of } x \text{ (left area)}$\\
$\Phi(z) = \text{ standard normal cumulative function}$
\begin{align*}
z &= \frac{x-\mu}{\sigma}\\
P(X < x) &= \Phi(z)\\
\ell &= \Phi(z)\\
z &= \Phi^{-1}(\ell)
\end{align*}

{\bf Bernoulli Distribution:}\\
$X \sim \Bern{p} $\\
$X = \text{ 0 for fail or 1 for success}$\\
$p = \text{ probability of success}$
\begin{align*}
P(X=0) ~&=~ 1-p \\
P(X=1) ~&=~ p \\
\mu ~&=~ p\\
\sigma ~&=~ \sqrt{p(1-p)}
\end{align*}

{\bf Geometric Distribution:}\\
$X \sim \Geo{p}$ \\
$X = \text{ number of trials until first success}$\\
$p = \text{ probability of success on each trial}$\\
$n = \text{ a possible number of trials}$
\begin{align*}
P(X=n) ~&=~ (1-p)^{n-1}(p)\\
\mu ~&=~ \frac{1}{p}\\
\sigma ~&=~ \sqrt{\frac{1-p}{p^2}} 
\end{align*}

{\bf Mean-Sampling Distribution:}\\
$\bar{X} = \text{ sample mean}$\\
$s = \text{ sample standard deviation}$\\
$n = \text{ sample size}$\\
$\mu = \text{ population mean}$\\
$\sigma = \text{ population standard deviation}$\\
$SE = \text{ standard error}$
$$SE = \frac{\sigma}{\sqrt{n}} $$
If $n\ge 30$ (or if population is normal) then:
$$\bar{X} \sim \N{\mu}{SE} $$

\columnbreak

{\bf Binomial Distribution:}\\
$X \sim \B{n}{p}$\\
$X = \text{ number of successes from $n$ trials} $\\
$p = \text{ probability of success on each trial}$\\
$n = \text{ number of trials}$\\
$k = \text{ a possible number of successes}$
\begin{align*}
P(X=k) ~&=~ {n \choose k} p^k (1-p)^{n-k}\\
\mu ~&=~ np\\
\sigma ~&=~ \sqrt{np(1-p)} 
\end{align*}
If $np \ge 10$ and $n(1-p) \ge 10$, then
$$X \sim \N{\mu}{\sigma} $$
Continuity correction:
$$P(X \le k) ~\approx~ \Phi\left(\frac{k+0.5-\mu}{\sigma}\right)$$



%{\bf Proportion-Sampling Distribution:}\\
%%$\hat{p} \sim \Sampp{n}{p}$\\
%$\hat{p} = \text{ sample proportion} $\\
%$p = \text{ population proportion}$\\
%$n = \text{ number of trials}$\\
%$y = \text{ a possible sample proportion}$
%\begin{align*}
%P\left( \hat{p}=y \right) ~&=~ {n \choose ny} p^{ny} (1-p)^{n-ny}\\
%\mu ~&=~ p\\
%\sigma ~&=~ \sqrt{\frac{p(1-p)}{n}} 
%\end{align*}
%If $np \ge 10$ and $n(1-p) \ge 10$, then
%$$\hat{p} \sim \N{\mu}{\sigma} $$
%Continuity correction:
%$$P(\hat{p} \le y) ~\approx~ \Phi\left(\frac{ny+0.5-\mu}{n\sigma}\right)$$
%$$P(\hat{p} < y) ~\approx~ \Phi\left(\frac{ny-0.5-\mu}{n\sigma}\right)$$



{\bf Confidence Interval:}\\
$CI = \text{ confidence interval}$\\
$\gamma = \text{ confidence level}$\\
$\bar{x} = \text{ sample mean}$ \\
$s = \text{ sample standard deviation}$ 
\begin{align*}
z^{\star} &= \Phi^{-1}\left(\frac{\gamma+1}{2} \right) \\
SE &\approx \frac{s}{\sqrt{n}}\\
CI &= \bar{x} \pm z^{\star} SE
\end{align*}

{\bf Hypothesis testing:}\\
$H_0:~~~\mu = \mu_0$\\
$H_A:~~~\mu \ne \mu_0$\\
$\bar{x} = \text{ a possible/specific/observed sample mean}$\\
$s = \text{ sample standard deviation}$\\
$\alpha = \text{ significance level} $
\begin{align*}
\sigma &\approx s\\
z &=  \frac{\bar{x}-\mu_0}{SE} \\\\
p\text{-value } &= P\left(\lvert Z \rvert > |z|  \right)\\
&= 2 \cdot \Phi\left( -|z| \right)
\end{align*}
If $p$-value $< \alpha$, then reject $H_0$, else retain $H_0$.

\end{multicols}



%\newpage
%
%\rhead{\textsc{Definitions and Formulas}}
%{\bf Sample statistics:}\vspace{-10pt}
%\begin{multicols}{2}\noindent
%$n=\text{sample size} $\\
%$x_i=\text{the $i$th value in a sample} $\\
%$\bar x = \text{sample mean}$\\
%$s = \text{sample standard deviation}$\\\\
%$\bar x = \cfrac{\sum_{i=1}^n x_i}{n}$
%
%\columnbreak \noindent
%$Q_1$ = first quartile\\
%$m$ = median\\
%$Q_3$ = third quartile\\
%IQR = inter-quartile range = $Q3-Q1$\\\\
%$s = \sqrt{\cfrac{\sum_{i=1}^n (x_i-\bar x)^2}{n-1}}$ 
%\end{multicols}
%
%{\bf Population parameters:}\\
%$\mu = \text{population mean}$\\
%$\sigma = \text{population standard deviation}$\\
%
%{\bf Probability:}\\
%$\Omega = \text{set of all possible equally likely outcomes}$\\
%$A = \text{event A, a set of outcomes}$\\
%$A^c = \text{The complement of }A$\\
%$B = \text{event B, another set of outcomes}$\\
%$|A| = \text{size of set, number of outcomes in } A$\\
%$P(A) = \text{probability of }A$\\
%$P(A \AND B) = \text{probability of both $A$ and $B$}$\\
%$P(A \OR B) = \text{probability of either $A$ or $B$ (or both)}$\\
%$P(A | B) = \text{probability of $A$ given $B$}$\\\\
%$P(A) = \cfrac{|A|}{|\Omega|}$\\\\
%$0 \le P(A) \le 1$\\
%$P(A \AND B) = P(A) \cdot P(B|A)$\\
%$P(A \OR B) = P(A) + P(B) - P(A\AND B)$\\
%$P(A^c) = 1 - P(A)$
%\\\\
%$A$, $B$ are disjoint (mutually exclusive) ~~$\iff~~P(A\AND B) = 0$\\
%$A$, $B$ are non-disjoint ~~$\iff~~P(A\AND B) > 0$\\
%$A$, $B$ are exhaustive ~~$\iff~~P(A\OR B) = 1$\\
%$A$, $B$ are complements ~~$\iff$~~ $A$, $B$ are disjoint and exhaustive ~~$\iff$~~ $B=A^c$\\
%$A$, $B$ are independent ~~$\iff~~P(A\AND B) = P(A)\times P(B) ~~\iff~~ P(A|B)=P(A)$\\
%
%{\bf Random variables and distributions:}\\
%$X=$ random variable \\
%$x_i=$ the $i$th possible value of $X$. (Notice different meaning here {vs.} sample statistics.)\\
%$k=$ number of possible values of $X$.\\
%$E(X)=\mu=$ expected value of $X$\\
%$\sigma=$ standard deviation of $X$\\
%$\mu = \sum_{i=1}^k  x_i \cdot P(X=x_i) $\\
%$\sigma = \sqrt{\sum_{i=1}^k (x_i-\mu)^2 \cdot P(X=x_i)}$


\newpage
\newcommand{\zlo}{z_\textsc{lower}}
\newcommand{\zhi}{z_\textsc{upper}}

\begin{questions}
\question[10] Let random variable $X$ be normally distributed with mean $\mu=50$ and standard deviation $\sigma=12$. What is the probability that $X$ is between 46 and 54?
\vfill
\question[10] Let random variable $\bar{X}$ be the sample mean of 36 draws from a normally distributed population with mean $\mu=50$ and standard deviation $\sigma=12$. What is the probability that $\bar{X}$ is between 46 and 54?
\vfill

\newpage

\question[10] Let random variable $X$ be normally distributed with mean $\mu=50$ and standard deviation $\sigma=12$. What is the $x$-score of the 80th percentile?
\vfill


\question[10] Let random variable $Y$ be normally distributed with mean $\mu=72$ and an unknown standard deviation $\sigma$. However, you know the 90th percentile is $y=79$. What is the distribution's standard deviation?
\vfill

\question[10] Let random variable $W$ be normally distributed with an unknown mean $\mu$ and standard deviation $\sigma=0.5$. However, you know the 30th percentile is $w=8$. What is the distribution's mean?
\vfill

\newpage

\question[10] Let each trial have a probability of success $p=0.61$. 
\begin{parts}
\part What is the probability that in 400 trials there are 250 successes?
\vfill
\part What is the probability that in 400 trials there are at least 250 successes? (Please use a normal approximation and continuity correction. Also, remember that $p=0.61$.)
\vfill
\end{parts}

\newpage

\question[10] A random sample of size $n=89$ has a mean $\bar{x}=23.4$ and a sample standard deviation $s=5.6$ (and no apparent skew). Determine a confidence interval of the population's mean using a confidence level of $75$\%.

\newpage

\question[10] A population is claimed to have a mean $\mu=678$. However, you are skeptical, so you decide you'll take a random sample and run a two-tail hypothesis test with a significance level $\alpha=0.05$.  

Your random sample of size $n=211$ results in a sample mean of $\bar{x}=664.4$ and a sample standard deviation $s=101.3$. What do you conclude?

\newpage
\question[10] What is a sampling distribution?

\end{questions}

\includepdf[pages=1]{ztable/ztable.pdf}
\includepdf[pages=2]{ztable/ztable.pdf}

\end{document}
