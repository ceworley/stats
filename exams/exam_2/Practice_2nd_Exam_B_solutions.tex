\documentclass[12pt,letterpaper,addpoints]{exam}
\usepackage[utf8]{inputenc}
\usepackage{amsmath}
\usepackage{amsfonts}
\usepackage{amssymb}
\usepackage{amsthm}
\usepackage{graphicx}
\usepackage{tabularx}
\usepackage[left=2cm,right=2cm,top=2cm,bottom=2cm]{geometry}
\usepackage{multicol}
\usepackage{multirow,array}
\usepackage{newtxtext,newtxmath}
\usepackage{lastpage}
\usepackage{enumitem}
\usepackage{pdfpages}
\newcolumntype{Y}{>{\centering\arraybackslash}X}
\firstpageheader{}{}{\includegraphics[scale=0.5]{BHCClogoBW.jpg}\hspace{-40pt}\vspace{-50pt}}
\firstpagefooter{}{}{Page \thepage ~of \pageref{LastPage}}
\runningheader{ \textsc{Math 181 2nd Exam Practice B }}{SOLUTIONS}{ \textsc{Spring 2019}}%{\includegraphics[scale=.5]{BHCClogoBW.jpg}\vspace{-10pt}}%\hspace{-60pt}\vspace{-10pt}}
\runningheadrule
\runningfooter{}{}{Page \thepage ~of \pageref{LastPage}}
\renewcommand{\thequestion}{{\bf Q\arabic{question}}}
%\renewcommand{\questionlabel}{{\thequestion .}}
%\pointformat{\fbox{\themarginpoints \,pt}}
%\pointsinrightmargin
%\setlength{\rightpointsmargin}{1.5cm}
%\pointsinmargin
%\setlength{\marginpointssep}{10pt}

\begin{document}
\printanswers

\newcommand{\AND}{~\textsc{and}~}
\newcommand{\OR}{~\textsc{or}~}

\begin{center}
\text{ }\\
\vspace{40pt}
\textsc{{\Huge Math 181 2nd Exam Practice B SOLUTIONS}}\\\vspace{5pt}
\textsc{{\large Spring 2019}}\\
\vspace{60pt}
\makebox[\textwidth]{\large Name:\enspace\hrulefill}\\
\vspace{40pt}
\fbox{\fbox{\begin{minipage}{6in}
\vspace{0.2in}
\begin{itemize}
\item Write your {\bf full name} on the line above.
\item Show your work. Incorrect answers with work can receive partial credit.
\item Attempt every question; showing you understand the question earns some credit.
\item If you run out of room for an answer, continue on the back of the page. Before doing so, write ``see back'' with a circle around it.
\item You can use 1 page (front and back) of notes.
\item You can use (and probably need) a calculator.
\item You can use the Geogebra Scientific Calculator instead of a calculator. You need to put your phone on {\bf airplane mode} and then within the application, start {\bf exam mode}; you should see a green bar with a timer counting up.
\item If a question is confusing or ambiguous, please ask for clarification; however, you will not be told how to  answer the question.
\item {\bf Box your final answer}.
\item A formula sheet is attached to this test.
\vspace{0.2in}
\end{itemize}
\end{minipage}
\begin{minipage}{0.2in}\text{ }\end{minipage}}}
\vfill
Do not write in this grade table.\\
\gradetable[h][questions]\\
\vfill
\end{center}

%%%%%%%%%%%%%%%%%%%%%%%%%%%%%%%%%%%%%%%%%%%%%%%%%%%%%%%%%%%%%%%%%%%%%%%%%%%%%%%
\newcommand{\N}[2]{\mathcal{N}\big(#1\,,~#2\big)}
\newcommand{\Bern}[1]{\texttt{Bern}\big(#1\big)}
\newcommand{\Geo}[1]{\texttt{Geo}\big(#1\big)}
\newcommand{\B}[2]{\mathcal{B}\big(#1\,,~#2\big)}
\newcommand{\Sampp}[2]{\texttt{Samp}_{\hat{p}}\big(#1,~#2\big)}
%\newcommand{\Sampx}[2]{\texttt{Samp}_{\hat{p}}\big(#1,~#2\big)}

\newpage

\begin{multicols}{2}

%\rhead{\textsc{Definitions and Formulas}}

{\bf Normal Distribution:}\\
$X \sim \N{\mu}{\sigma}$\\
$\mu = \text{ population mean} $\\
$\sigma = \text{ population standard deviation} $\\
$x = \text{ possible value of }X $\\
$\ell = \text{ percentile of } x \text{ (left area)}$\\
$\Phi(z) = \text{ standard normal cumulative function}$
\begin{align*}
z &= \frac{x-\mu}{\sigma}\\
P(X < x) &= \Phi(z)\\
\ell &= \Phi(z)\\
z &= \Phi^{-1}(\ell)
\end{align*}

{\bf Bernoulli Distribution:}\\
$X \sim \Bern{p} $\\
$X = \text{ 0 for fail or 1 for success}$\\
$p = \text{ probability of success}$
\begin{align*}
P(X=0) ~&=~ 1-p \\
P(X=1) ~&=~ p \\
\mu ~&=~ p\\
\sigma ~&=~ \sqrt{p(1-p)}
\end{align*}

{\bf Geometric Distribution:}\\
$X \sim \Geo{p}$ \\
$X = \text{ number of trials until first success}$\\
$p = \text{ probability of success on each trial}$\\
$n = \text{ a possible number of trials}$
\begin{align*}
P(X=n) ~&=~ (1-p)^{n-1}(p)\\
\mu ~&=~ \frac{1}{p}\\
\sigma ~&=~ \sqrt{\frac{1-p}{p^2}} 
\end{align*}

{\bf Mean-Sampling Distribution:}\\
$\bar{X} = \text{ sample mean}$\\
$s = \text{ sample standard deviation}$\\
$n = \text{ sample size}$\\
$\mu = \text{ population mean}$\\
$\sigma = \text{ population standard deviation}$\\
$SE = \text{ standard error}$
$$SE = \frac{\sigma}{\sqrt{n}} $$
If $n\ge 30$ (or if population is normal) then:
$$\bar{X} \sim \N{\mu}{SE} $$

\columnbreak

{\bf Binomial Distribution:}\\
$X \sim \B{n}{p}$\\
$X = \text{ number of successes from $n$ trials} $\\
$p = \text{ probability of success on each trial}$\\
$n = \text{ number of trials}$\\
$k = \text{ a possible number of successes}$
\begin{align*}
P(X=k) ~&=~ {n \choose k} p^k (1-p)^{n-k}\\
\mu ~&=~ np\\
\sigma ~&=~ \sqrt{np(1-p)} 
\end{align*}
If $np \ge 10$ and $n(1-p) \ge 10$, then
$$X \sim \N{\mu}{\sigma} $$
Continuity correction:
$$P(X \le k) ~\approx~ \Phi\left(\frac{k+0.5-\mu}{\sigma}\right)$$



%{\bf Proportion-Sampling Distribution:}\\
%%$\hat{p} \sim \Sampp{n}{p}$\\
%$\hat{p} = \text{ sample proportion} $\\
%$p = \text{ population proportion}$\\
%$n = \text{ number of trials}$\\
%$y = \text{ a possible sample proportion}$
%\begin{align*}
%P\left( \hat{p}=y \right) ~&=~ {n \choose ny} p^{ny} (1-p)^{n-ny}\\
%\mu ~&=~ p\\
%\sigma ~&=~ \sqrt{\frac{p(1-p)}{n}} 
%\end{align*}
%If $np \ge 10$ and $n(1-p) \ge 10$, then
%$$\hat{p} \sim \N{\mu}{\sigma} $$
%Continuity correction:
%$$P(\hat{p} \le y) ~\approx~ \Phi\left(\frac{ny+0.5-\mu}{n\sigma}\right)$$
%$$P(\hat{p} < y) ~\approx~ \Phi\left(\frac{ny-0.5-\mu}{n\sigma}\right)$$



{\bf Confidence Interval:}\\
$CI = \text{ confidence interval}$\\
$\gamma = \text{ confidence level}$\\
$\bar{x} = \text{ sample mean}$ \\
$s = \text{ sample standard deviation}$ 
\begin{align*}
z^{\star} &= \Phi^{-1}\left(\frac{\gamma+1}{2} \right) \\
SE &\approx \frac{s}{\sqrt{n}}\\
CI &= \bar{x} \pm z^{\star} SE
\end{align*}

{\bf Hypothesis testing:}\\
$H_0:~~~\mu = \mu_0$\\
$H_A:~~~\mu \ne \mu_0$\\
$\bar{x} = \text{ a possible/specific/observed sample mean}$\\
$s = \text{ sample standard deviation}$\\
$\alpha = \text{ significance level} $
\begin{align*}
\sigma &\approx s\\
z &= \left| \frac{\bar{x}-\mu_0}{SE} \right|\\\\
p\text{-value } &= P\left(\lvert Z \rvert > z  \right)\\
&= 2 \Phi\left( -z \right)
\end{align*}
If $p$-value $< \alpha$, then reject $H_0$, else retain $H_0$.

\end{multicols}



%\newpage
%
%\rhead{\textsc{Definitions and Formulas}}
%{\bf Sample statistics:}\vspace{-10pt}
%\begin{multicols}{2}\noindent
%$n=\text{sample size} $\\
%$x_i=\text{the $i$th value in a sample} $\\
%$\bar x = \text{sample mean}$\\
%$s = \text{sample standard deviation}$\\\\
%$\bar x = \cfrac{\sum_{i=1}^n x_i}{n}$
%
%\columnbreak \noindent
%$Q_1$ = first quartile\\
%$m$ = median\\
%$Q_3$ = third quartile\\
%IQR = inter-quartile range = $Q3-Q1$\\\\
%$s = \sqrt{\cfrac{\sum_{i=1}^n (x_i-\bar x)^2}{n-1}}$ 
%\end{multicols}
%
%{\bf Population parameters:}\\
%$\mu = \text{population mean}$\\
%$\sigma = \text{population standard deviation}$\\
%
%{\bf Probability:}\\
%$\Omega = \text{set of all possible equally likely outcomes}$\\
%$A = \text{event A, a set of outcomes}$\\
%$A^c = \text{The complement of }A$\\
%$B = \text{event B, another set of outcomes}$\\
%$|A| = \text{size of set, number of outcomes in } A$\\
%$P(A) = \text{probability of }A$\\
%$P(A \AND B) = \text{probability of both $A$ and $B$}$\\
%$P(A \OR B) = \text{probability of either $A$ or $B$ (or both)}$\\
%$P(A | B) = \text{probability of $A$ given $B$}$\\\\
%$P(A) = \cfrac{|A|}{|\Omega|}$\\\\
%$0 \le P(A) \le 1$\\
%$P(A \AND B) = P(A) \cdot P(B|A)$\\
%$P(A \OR B) = P(A) + P(B) - P(A\AND B)$\\
%$P(A^c) = 1 - P(A)$
%\\\\
%$A$, $B$ are disjoint (mutually exclusive) ~~$\iff~~P(A\AND B) = 0$\\
%$A$, $B$ are non-disjoint ~~$\iff~~P(A\AND B) > 0$\\
%$A$, $B$ are exhaustive ~~$\iff~~P(A\OR B) = 1$\\
%$A$, $B$ are complements ~~$\iff$~~ $A$, $B$ are disjoint and exhaustive ~~$\iff$~~ $B=A^c$\\
%$A$, $B$ are independent ~~$\iff~~P(A\AND B) = P(A)\times P(B) ~~\iff~~ P(A|B)=P(A)$\\
%
%{\bf Random variables and distributions:}\\
%$X=$ random variable \\
%$x_i=$ the $i$th possible value of $X$. (Notice different meaning here {vs.} sample statistics.)\\
%$k=$ number of possible values of $X$.\\
%$E(X)=\mu=$ expected value of $X$\\
%$\sigma=$ standard deviation of $X$\\
%$\mu = \sum_{i=1}^k  x_i \cdot P(X=x_i) $\\
%$\sigma = \sqrt{\sum_{i=1}^k (x_i-\mu)^2 \cdot P(X=x_i)}$


\newpage
\newcommand{\zlo}{z_\textsc{lower}}
\newcommand{\zhi}{z_\textsc{upper}}

\begin{questions}
\question[10] Hannah is curious about the expected number of rolls of a 6-sided die before getting every side, but Hannah forgets how to analyze it mathematically. So, she gets a 6-sided die and rolls it until she sees every number and writes down how many rolls it took. She repeats this over and over, getting the following sample:
\begin{center}
\begin{tabular}{*{15}{c}}
18 & 19 & 26 & 23 & 17 & 12 & 11 & 12 & 23 & 16 & 13 & 8 & 10 & 8 & 7 \\
 19 & 14 & 11 & 15 & 17 & 20 & 24 & 12 & 18 & 10 & 9 & 22 & 24 & 14 & 14
\end{tabular}
\end{center}
Hannah determines the sample size $n=30$, sample mean $\bar{x}=15.53$, and sample standard deviation $s=5.41$.
\begin{parts}
\part Determine a 92\% confidence interval for the expected number of rolls to get all sides. You can assume the sampling distribution is normal (even though the population is not normal).
\begin{solution}
We need to consider the sampling distribution. Because $n\ge30$, we think the sampling distribution is approximately normal. We need standard error for sampling distributions.
$$SE = \frac{5.41}{\sqrt{30}} = 0.9877$$
%$$\bar{X} \sim \N{15.53}{0.9877} $$
We draw a sketch, where $ME$ is the margin of error, defined as $ME=z^\star SE$.
\begin{center}
\includegraphics[scale=0.8]{figures/sketch5.pdf}
\end{center}
We can find $z^\star$ by recognizing it's left area is 0.96. In other words, $\Phi(z^\star)=0.96$. In other words, $P(Z<z^\star)=0.96$. In other words, $\ell=0.96$. One way or another, we need to use the $z$ table backwards to find $z^\star$.
$$z^\star ~~=~~ \Phi^{-1}(0.96) ~~=~~ 1.75  $$
We now can create a confidence interval.
\begin{align*}
CI ~&=~ \bar{x} \pm ME \\
&=~ \bar{x} \pm z^\star SE \\
&=~ 15.53 \pm (1.75) (0.9877) \\
&=~ (13.8\,,~17.3)
\end{align*}
\end{solution}
\part After this study, would you believe a friend that suggests the expected number of rolls is 11? Why or why not?
\begin{solution}
Nope. 11 is outside the confidence interval. In fact, 11 would have a $z$-score of $-4.59$, which makes it quite inconsistent with our study.
\end{solution}
\end{parts}

\newpage

\question[10] Imagine each trial has a 42\% chance of success.  Let random variable $W$ represent the result of a trial, where 0 means failure and 1 means success.
\begin{parts}
\part What is the standard deviation of $W$?
\begin{solution}
A single trial follows a Bernoulli distribution.
$$\sigma = \sqrt{p(1-p)} = \sqrt{(0.42)(0.58)} = \fbox{0.494} $$
\end{solution}
\vfill
\part What is the expected number of trials until getting a success?
\begin{solution}
Trials until success follows a geometric distribution.
$$\mu = \frac{1}{p} = \frac{1}{0.42} = \fbox{2.38} $$
\end{solution}
\vfill
\part What is the standard deviation of number of trials until getting a success?
\begin{solution}
Trials until success follows a geometric distribution.
$$\sigma = \sqrt{\frac{1-p}{p^2}} = \sqrt{\frac{0.58}{(0.42)^2}} = \fbox{1.81}$$
\end{solution}
\vfill
\part What is the probability of getting 30 successes from 75 trials?
\begin{solution}
Number of successes in $n$ trials follows a binomial distribution.
$$P(X=30)~~=~~{75 \choose 30}(0.42)^{30}(0.58)^{45}$$
$$P(X=30)~~=~~\fbox{0.088}$$
\end{solution}
\vfill
\end{parts}


\newpage

\question[10] If each trial has a 33\% chance of success and there are 200 trials, what is the probability that the number of successes is more than 67? Please use a normal approximation with the continuity correction.
\begin{solution}
We determine the mean and standard deviation of the binomial distribution.
\begin{align*}
\mu &= np \\
&= (200)(0.33) \\
&= 66
\end{align*}
\begin{align*}
\sigma &= \sqrt{np(1-p)} \\
&= \sqrt{(200)(0.33)(0.67)} \\
&= 6.65
\end{align*}
A sketch is helpful... your sketch can be simpler. If you look closely, the continuity correction should be clear.
\begin{center}
\includegraphics[scale=1]{figures/sketch6.pdf}
\end{center}
In order to estimate the area of the bars higher than 67, we find the area under the curve when $x>67.5$. That extra 0.5 is the continuity correction, necessary because these bars have width of 1 and are centered on the integers. We find the $z$-score.
$$z = \frac{67.5-66}{6.65} = 0.23 $$
We find the right area.
\begin{align*}
P(X>67) ~~&\approx~~ P(Z>0.23) \\
&=~~ 1-\Phi(0.23) \\
&=~~ 1-0.5910 \\
&=~~ \fbox{0.4090}
\end{align*}
\end{solution}



\newpage

\question[10] Perform a two-tail hypothesis test with $\mu_0=100$, $n=50$, $\bar{x}=103.2$, $s=14.4$, and $\alpha=0.10$.
\begin{solution}
State the hypotheses.
$$H_0:~~ \mu=100 $$
$$H_A:~~ \mu\ne 100 $$
Calculate the standard error of the null's sampling distribution by assuming $\sigma \approx s$.
$$SE ~=~ \frac{14.4}{\sqrt{50}} ~=~ 2.036468 $$
Find a $z$-score.
$$z = \frac{103.2-100}{2.036} = 1.57 $$
Draw a sketch of the null's sampling distribution.
\begin{center}
\includegraphics[scale=1]{figures/sketch7.pdf}
\end{center}
We calculate the $p$-value.
\begin{align*}
p\text{-value} ~~&=~~ 2 \cdot \Phi(-1.57) \\
&=~~ (2)(0.0582) \\
&=~~ 0.1164
\end{align*}
Compare $p$-value to $\alpha$.
\begin{align*}
0.1164 &> 0.10 \\
p\text{-value} &> \alpha
\end{align*}
So, we \fbox{retain the null.}
\end{solution}



\newpage

\question[10] Let $X\sim\N{500}{20}$.
\begin{parts}
\part Evaluate $P(470 < X < 520)$.
\begin{solution}
Find the $z$-scores.
$$\zlo = \frac{470-500}{20} = -1.5 $$
$$\zhi = \frac{520-500}{20} = 1 $$
Make a sketch.
\begin{center}
\includegraphics[scale=0.8]{figures/sketch8.pdf}
\end{center}
Calculate the probability.
\begin{align*}
P(470<X<520) ~~&=~~ \Phi(1) - \Phi(-1.5) \\
&=~~ 0.8413-0.0668 \\
&=~~ \fbox{0.7745}
\end{align*}
\end{solution}

\part Determine $x$ such that $P(X > x) ~=~ 0.40$.
\begin{solution}
Draw a sketch.
\begin{center}
\includegraphics[scale=0.8]{figures/sketch9.pdf}
\end{center}
We want the horizontal coordinate of the boundary. Find the $z$.
$$z ~=~ \Phi^{-1}(0.6) ~=~ 0.25$$
Find $x$.
$$x ~~=~~ \mu + z\sigma ~~=~~ 500+(0.25)(20) ~~=~~ \fbox{505}  $$
\end{solution}
\end{parts}

\newpage

\question[10] There is a continuous population with $\mu=500$ and $\sigma=20$. What is the probability that a sample of size 100 has a mean between 500 and 503?
\begin{solution}
Calculate the standard error.
$$SE = \frac{20}{\sqrt{100}} = 2 $$
Find the $z$-scores.
$$\zlo = \frac{500-500}{2} = 0 $$
$$\zhi = \frac{503-500}{2} = 1.5 $$
Sketch the sampling distribution with the shaded region.
\begin{center}
\includegraphics[scale=1]{figures/sketch10.pdf}
\end{center}
Calculate the probability.
\begin{align*}
P(500<\bar{X}<503) ~~&=~~ \Phi(1.5) - \Phi(0) \\
&=~~ 0.9332-0.5000 \\
&=~~ \fbox{0.4332}
\end{align*}

\end{solution}

\end{questions}

\includepdf[pages=1]{ztable/ztable.pdf}
\includepdf[pages=2]{ztable/ztable.pdf}

\end{document}
