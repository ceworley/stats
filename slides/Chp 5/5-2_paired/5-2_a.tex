%%%%%%%%%%%%%%%%%%%%%%%%%%%%%%%%%%%%
% Slide options
%%%%%%%%%%%%%%%%%%%%%%%%%%%%%%%%%%%%

% Option 1: Slides with solutions

%\documentclass[slidestop,compress,mathserif]{beamer}
%\newcommand{\soln}[1]{#1}
%\newcommand{\solnGr}[1]{#1}

% Option 2: Handouts without solutions

\documentclass[11pt,containsverbatim,handout]{beamer}
\newcommand{\soln}[1]{ }
\newcommand{\solnGr}[1]{ }


\title[Chp 5.2: Paired data]{Chp 5.2: Paired data}

\begin{document}

\section{Paired Data}
\begin{frame}
\frametitle{Paired Data}
Paired data often arise when measuring the same individuals twice (before and after a period). \pause
\begin{center}
\begin{tabular}{|c|c|c|c|} \hline
Individual & Weight in 2010 & Weight in 2020 & Diff \\ \hline
Marion & 140 & 135 & -5\\
Sylvester & 190 & 249 & 59\\
Florence & 183 & 183 & 0\\
David & 90  & 134 & 44\\
Gertrude & 208 & 180 & -28\\
\vdots&\vdots&\vdots & \vdots \\ \hline
\end{tabular}
\end{center}
What would an implied question be? \pause
\begin{center}
\soln{Do individual humans tend to gain weight over time?}
\vfill
\end{center} \pause
Two sets of observations are paired if each observation in one set has a special correspondence or connection with exactly one observation in the other data set.
\end{frame}

\section{Unpaired Data}
\begin{frame}
\frametitle{Unpaired Data}
Two separate random samples would produce unpaired data.
\begin{center}
\begin{tabular}{|c|c|c|c|c|} \cline{1-2} \cline{4-5}
\multicolumn{2}{|c|}{year=2010} && \multicolumn{2}{|c|}{year=2020} \\ 
Individual & Weight && Individual& Weight \\ \cline{1-2} \cline{4-5}
Lonzo   & 140   && Henry & 310\\
Rosalia & 190   && Harvey & 250\\
Leora   & 183   && Phoebe & 210\\
Otis    & 90    && Donna & 150\\
Edward  & 208   && John & 110\\
\vdots&\vdots && \vdots & \vdots \\\cline{1-2} \cline{4-5}
\end{tabular}
\end{center}
What would an implied question be? \pause
\begin{center}
\soln{Did humans (as a species) tend to gain weight over time?}
\vfill
\end{center} \pause
We will discuss unpaired analysis in Chapter 5.3 (next class).

With paired data, we consider a {\bf mean of differences}.

With unpaired data, we consider a {\bf difference of means}.
\end{frame}

\newcommand{\Xd}{X_\text{diff}}
\newcommand{\xd}{x_\text{diff}}

\section{Derivation of formulas}
\begin{frame}
\frametitle{Derivation of paired formulas}
Let random variable $D_i$ represent the (unknown) difference from a (yet to be) randomly selected individual $i$. \pause

We want to predict what happens when we find a mean of differences.
$$\bar{D} = \frac{D_1+D_2+D_3+...+D_n}{n}$$ \pause

The central limit theorem still applies! \pause
\vfill

As $n\to \infty$, $\bar{D}$ becomes normally distributed.\pause
\vfill 

Basically, we can treat these differences just like any other independent and identically distributed random variables. 
\vfill
\end{frame}


\section{Note about notation}
\begin{frame}
\frametitle{Note about notation}
\begin{itemize}
\item I used $\bar{D}$ for the random variable representing an unknown mean of differences. \pause
\item I would use $\bar{d}$ for a specific (observed, critical, etc) mean of difference. \pause
\item The book uses $\bar{x}_\text{diff}$ for both of these concepts. This is misleading, as it looks like a difference of means, not a mean of differences. \pause
\item I would at least prefer using $\overline{X_\text{diff}}$ and $\overline{x_\text{diff}}$ to emphasize we are finding a mean of differences. \pause
\item The book's notation of $\mu_\text{diff}$ (for the population's true difference) is useful. We could also use $E(D)$ or $\mu_{D}$.
\item In order to match the book as much as possible, I will now use $x_{\text{diff},i}$ and $\overline{X_\text{diff}}$ and $\overline{x_\text{diff}}$ and $\mu_\text{diff}$.
\end{itemize}
\end{frame}

\newcommand{\xdi}{x_{\text{diff},i}}
\newcommand{\bxd}{\overline{x_\text{diff}}}
\newcommand{\bXd}{\overline{X_\text{diff}}}
\newcommand{\mud}{\mu_\text{diff}}

\section{Example problem}
\begin{frame}
\frametitle{Example problem}
A teacher wonders if, on average, a random student will perform about the same on two exams. She decides to run a two-tail $t$ test on a random sample of size $n=5$ with a signficance level $\alpha=0.05$.

\pause

Here are the results of her study:
\begin{center}
\begin{tabular}{|c|c|c|} \hline
Student & Exam 1 & Exam 2 \\ \hline
Norma   & 98     & 96     \\  
Elliot  & 15     & 10     \\  
Walton  & 61     & 61     \\  
Mable   & 80     & 79     \\  
Loretta & 10     & 8      \\  \hline
\end{tabular}
\end{center}
Perform the $t$ test.
\end{frame}

\section{Example problem solution}
\begin{frame}
\frametitle{Example problem solution}
Find the differences. \pause
\begin{center}
\begin{tabular}{|c|c|c|c|} \hline
$i$    & $x_{1,i}$ & $x_{2,i}$ & $x_{\text{diff},i}$ \\ \hline
1   & 98     & 96     & -2\\  
2   & 15     & 10     & -5\\  
3   & 61     & 61     &  0\\  
4   & 80     & 79     & -1\\  
5   & 10     & 8      & -2\\  \hline
\end{tabular}
\end{center} \pause
Find the (differences') sample mean. \pause
$$\bxd = \frac{\sum\limits_{i=1}^n \xdi}{n} = \frac{-2-5+0-1-2}{5} = -2 $$ \pause
Find the (differences') sample standard deviation. \pause
$$s = \sqrt{\frac{\sum\limits_{i=1}^n (\xdi-\bxd)^2}{n-1}} = \sqrt{\frac{(0)^2+(3)^2+(2)^2+(1)^2+(0)^2}{5-1}} = 1.87 $$
\end{frame}


\begin{frame}
We are doing a two-tail test with the following:
\begin{align*}
n &=5 & \bxd &= -2 & s&= 1.87 & \alpha &= 0.05
\end{align*} \pause
State the hypotheses. \pause
\begin{align*}
H_0&:~~ \mud = 0 & H_A&:~~ \mud \ne 0
\end{align*} \pause
Determine the critical value, $t^\star$, such that $P(|T|>t^\star) = 0.05$. \pause
$$t^\star = 2.78 $$ \pause
Find the standard error (the standard deviation of the differences' sampling distribution). \pause
$$SE = \frac{s}{\sqrt{n}} = \frac{1.87}{\sqrt{5}} = 0.837 $$ \pause
Calculate an observed $t$ score. \pause
$$t_\text{obs} = \frac{(-2)-0}{0.837} = -2.39 $$
\end{frame}

\begin{frame}
From the previous slides:
\begin{align*}
n &=5 & \bxd &= -2 & s&= 1.87 & \alpha &= 0.05 \\
t^\star &= 2.78 & SE &= 0.837 & t_\text{obs} &= -2.39
\end{align*} \pause
We can determine a $p$-value. Remember we are doing a two-tail test, so $p$-value $= P(|T|>2.39)$.\pause
$$0.05 ~~<~~ p\text{-value} ~~<~~ 0.1  $$\pause
We can compare $t_\text{obs}$ and $t^\star$. We can also compare $p$-value and $\alpha$.\pause
$$|t_\text{obs}| < |t^\star| $$\pause
$$p\text{-value} > \alpha $$\pause
Thus, we retain the null hypothesis. \pause
\begin{center}
We maintain that maybe students do equally well on both tests.
\end{center}
\end{frame}


\section{Practice}
\begin{frame}
\frametitle{Practice}
The following table has paired data. Test the hypotheses of whether or not the differences have a population average of 0. Use $\alpha=0.1$. 
\begin{center}
\begin{tabular}{|c|c|c|} \hline
$i$    & $x_{1,i}$ & $x_{2,i}$ \\ \hline
1   & 50     & 54  \\  
2   & 23     & 25  \\  
3   & 96     & 97  \\  
4   & 47     & 49  \\  
5   & 10     & 16   \\  \hline
\end{tabular}
\end{center} \pause
\end{frame}
\vfill


\end{document}
