\documentclass[12pt,letterpaper]{article}
\usepackage[utf8]{inputenc}
\usepackage{amsmath}
\usepackage{amsfonts}
\usepackage{amssymb}
\usepackage{amsthm}
\usepackage{graphicx}
\usepackage{tabularx}
\usepackage[left=2cm,right=2cm,top=2cm,bottom=2cm]{geometry}
\usepackage{multicol}
\usepackage{lastpage}
\usepackage{fancyhdr}
\usepackage{multirow,array}
\usepackage{newtxtext,newtxmath}
\usepackage{lastpage}
\usepackage{enumitem}
\newcolumntype{Y}{>{\centering\arraybackslash}X}
\pagestyle{fancy}
\fancyhf{}
\lhead{\textsc{BHCC Mat-181}}
\chead{\textsc{Answers}}
\rhead{\textsc{HW Exercises 1.3-1.8}}
\rfoot{Page \thepage ~of \pageref{LastPage}}
\setenumerate[1]{label={\bf 1.\theenumi: }}
\setenumerate[2]{label={\bf (\theenumii): }}


\begin{document}
\begin{enumerate}
\setcounter{enumi}{8}
\item
\begin{enumerate}
\item The population of interest is all human births. The sample is 143,196 births in Southern California.
\item The sample is not random or representative, so generalization is difficult or impossible. The study was merely an observational study, so causal relationships can {\bf not} be established. 
\end{enumerate}

\item
\begin{enumerate}
\item The population of interest is all children between 5 and 15. The sample is 160 children between 5 and 15.
\item The sample is probably not random or representative, so generalization is difficult or impossible. The study was an experiment, so causal relationships can be established. 
\end{enumerate}

\item
\begin{enumerate}
\item The population of interest is people with asthma. The sample is 600 asthma patients aged 18-69 who rely on medication.
\item The sample is probably not random or representative, so generalization is difficult or impossible. The study was an experiment, so causal relationships can be established. 
\end{enumerate}

\item
\begin{enumerate}
\item The population of interest is all people. The sample is 129 UC Berkeley undergraduates (probably psychology students getting credit for participation).
\item The sample is probably not random or representative, so generalization is difficult or impossible. The study was an observational study, so causal relationships can {\bf not} be established. 
\end{enumerate}

\item
\begin{enumerate}
\item case (observational unit) %why not stick to ``case''???
\item variable
\item sample statistic
\item population parameter
\end{enumerate}

\item
\begin{enumerate}
\item case (observational unit) %why not stick to ``case''???
\item variable
\item sample statistic
\item population parameter
\end{enumerate}

\item
\begin{enumerate}
\item The explanatory variable might be study hours per week. The response variable might be GPA. But really, this language should be reserved for clearly causal relationships. This study is not an experiment, so we should be hesitant to use this language. It is easy for me to imagine students who have higher GPAs finding studying less stressful and therefor study more.
\item There seems to be a positive association. The GPA over 4.0 seems to be an error? Also, nobody should study 60 hours a week.
\item Observational study.
\item No. 
\end{enumerate}

\item
\begin{enumerate}
\item Again, I really wish we would reserve this language for experiments. But, by default the horizontal axis is the explanatory variable and the vertical axis is the response variable, so percent with bachelor's degree is explanatory and per capita income is response.
\item There is a clear positive association between the two variables.
\item No. This is an observational study, so we can not establish causal relationships.
\end{enumerate}





\end{enumerate}
\end{document}
