\documentclass[12pt,letterpaper]{article}
\usepackage[utf8]{inputenc}
\usepackage{amsmath}
\usepackage{amsfonts}
\usepackage{amssymb}
\usepackage{amsthm}
\usepackage{graphicx}
\usepackage{tabularx}
\usepackage[left=2cm,right=2cm,top=2cm,bottom=2cm]{geometry}
\usepackage{multicol}
\usepackage{lastpage}
\usepackage{fancyhdr}
\usepackage{multirow,array}
\usepackage{newtxtext,newtxmath}
\usepackage{lastpage}
\usepackage{enumitem}
\newcolumntype{Y}{>{\centering\arraybackslash}X}
\pagestyle{fancy}
\fancyhf{}
\lhead{\textsc{BHCC Mat-181}}
\chead{\textsc{Answers}}
\rhead{\textsc{HW Exercises 1.1-1.2}}
\rfoot{Page \thepage ~of \pageref{LastPage}}

\begin{document}
\begin{description}
\item[Exercise 1.1]
\begin{description}
\item[(a)] In the treatment group, the percent of pain-free patients was $\frac{10}{43}\approx 23\%$. In the control group, $\frac{2}{46} \approx 4\%$.
\item[(b)] At first glance, it seems acupuncture in area M is more effective than acupuncture in area S. However, neither method alleviated more than 25\% of the migraines. I wonder about alternative methods like medicine, meditation, or just waiting. In fact, I have all sorts of questions about the research, but unfortunately the paper is behind a paywall.
\item[(c)] My instinct says there is a significant difference between treatment and control groups. Of course, there is always the possibility the observed difference is just due to chance, but that possibility seems unlikely in this case.

I will say that the provided data could also suggest the control acupuncture (in area S) actually just made the patients worse. But, presumably there is evidence that acupuncture in S is similar to no acupuncture because the author called that group the control group.
\end{description}

\item[Exercise 1.2]
\begin{description}
\item[(a)] In the treatment group: $\frac{66}{85}\approx 78\%$. In the control group: $\frac{65}{81} \approx 80\%$.
\item[(b)] Well, $78\% < 80\%$, so the control group has slightly better outcomes.
\item[(c)] It seems highly plausible that this difference could be due to chance. 

For an analogy, imagine flipping a quarter 100 times and a penny 100 times. Imagine the quarter landed heads 53 times and the penny landed heads 49 times. Would you conclude the quarter has a higher probability of landing heads? No. This difference is probably due to natural variation.
\end{description}
\end{description}
\end{document}
