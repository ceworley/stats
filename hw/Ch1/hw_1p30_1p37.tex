\documentclass[12pt,letterpaper]{article}
\usepackage[utf8]{inputenc}
\usepackage{amsmath}
\usepackage{amsfonts}
\usepackage{amssymb}
\usepackage{amsthm}
\usepackage{graphicx}

\usepackage{hyperref}
\hypersetup{
    colorlinks=true,
    linkcolor=blue,
    filecolor=magenta,      
    urlcolor=cyan,
}
\urlstyle{same}

\usepackage{tabularx}
\usepackage[left=2cm,right=2cm,top=2cm,bottom=2cm]{geometry}
\usepackage{fancyhdr}
\usepackage{multicol}
\usepackage{multirow,array}
\usepackage{newtxtext,newtxmath}
\usepackage{relsize}
\usepackage{lastpage}
\usepackage{enumitem}
\usepackage{adjustbox}
\newcolumntype{Y}{>{\centering\arraybackslash}X}
\pagestyle{fancy}
\fancyhf{}
\lhead{\textsc{BHCC Mat-181}}
\chead{\textsc{Answers}}
\rhead{\textsc{Exercises 1.30-1.37}}
\rfoot{Page \thepage ~of \pageref*{LastPage}}
\setenumerate[1]{label={\bf 1.\theenumi: }}
\setenumerate[2]{label={\bf (\theenumii): }}

\begin{document}

\begin{enumerate}
\setcounter{enumi}{29}
\item \begin{enumerate}
\item Experiment
\item Yes
\end{enumerate}

\item \begin{enumerate}
\item The response variable is exam performance.
\item The explanatory variable is light level. Its levels are fluorescent, yellow, and desk.
\item The blocking variable is gender.
\end{enumerate}

\item \begin{enumerate}
\item This was an experiment because there was assignment: the researchers actively set a variable.
\item The explanatory variable is amount of vitamin C (and additives). The response variables were duration and severity of sickness and duration of symptoms.
\item The blindedness of the patients is dependent on how well the nurses kept the knowledge secret.
\item The study was double-blind if the nurses kept the knowledge secret.
\item This can introduce confounding variables. In this case, I wonder if the high doses of vitamin C were tasted by participants, and they either recognized the taste or did not like the sourness. Maybe the placebo tasted better, and therefor the participants believed in it more.
\end{enumerate}

\item \begin{enumerate}
\item The response variable is exam performance.
\item There are two factors (variables the researcher controls): light and noise treatments. Light has three levels: fluorescent, yellow, and desk. Noise has three levels: none, construction, and human.
\item Sex is a blocking variable.
\end{enumerate}

\item Get many participants. Assign each participant to study with instrumental, vocal, or no music. Measure how each participant does on a test.

\item Each student will drink 2 unmarked sodas, given in random order. After each drink, a blinded researcher will ask the student to rate the taste from 1 to 10.

\item \begin{enumerate}
\item Experiment
\item Treatment = exercise. Control = no exercise.
\item Yes. The blocking variable is age.
\item Not really. If you are told explicitly to not exercise, you probably can deduce the experiment is about exercise and that you are in the control group.
\item As long as the blindedness is not an issue, this can show causal relationships (experiment) and it can be generalized (representative sample).
\item I would worry about the harm of asking people to not exercise.
\end{enumerate}

\item \begin{enumerate}
\item Experiment
\item Treatment = chia seeds. Control = placebo.
\item Yes. The blocking variable is sex.
\item Yes. Placebo is a good blinding control. 
\item This can show causal relationships (experiment) and it can not be generalized (not a representative sample, volunteers).

\end{enumerate}

\end{enumerate}
\end{document}