\documentclass[12pt,letterpaper]{article}
\usepackage[utf8]{inputenc}
\usepackage{amsmath}
\usepackage{amsfonts}
\usepackage{amssymb}
\usepackage{amsthm}
\usepackage{graphicx}
\usepackage{tabularx}
\usepackage[left=2cm,right=2cm,top=2cm,bottom=2cm]{geometry}
\usepackage{multicol}
\usepackage{lastpage}
\usepackage{fancyhdr}
\usepackage{multirow,array}
\usepackage{newtxtext,newtxmath}
\usepackage{lastpage}
\usepackage{enumitem}
\newcolumntype{Y}{>{\centering\arraybackslash}X}
\pagestyle{fancy}
\fancyhf{}
\lhead{\textsc{BHCC Mat-181}}
\chead{\textsc{Answers}}
\rhead{\textsc{HW Exercises 2.27-2.35}}
\rfoot{Page \thepage ~of \pageref{LastPage}}
\setenumerate[1]{label={\bf 2.\theenumi: }}
\setenumerate[2]{label={\bf (\theenumii): }}
\setenumerate[3]{label={\bf \theenumiii: }}

\begin{document}
\newcommand{\AND}{\textsc{~and~}}
\newcommand{\OR}{\textsc{~or~}}
\newcommand{\GIVEN}{\textsc{~given~}}

\begin{enumerate}
\setcounter{enumi}{26}
\item\begin{enumerate}
\item $P(\text{1st is blue})=\frac{3}{10} = 0.3$
\item $P(\text{2nd is blue}\GIVEN \text{1st is blue... with replacement} )=\frac{3}{10} = 0.3$
\item $P(\text{2nd is blue}\GIVEN \text{1st is orange... with replacement} )=\frac{3}{10} = 0.3$
\item $P(\text{1st is blue} \AND \text{2nd is blue... with replacement} )= 0.3^2 = 0.09$
\item When drawing with replacement, the draws are independent. The probabilities of the second draw do not change based on the result of the first draw.
\end{enumerate}

\item \begin{enumerate}
\item $\frac{4}{12} \times \frac{3}{11} \approx 0.0909 $
\item $\frac{7}{12} \times \frac{6}{11} = 0.318 $
\item We first calculate the probability of the \emph{complement}.
$$P(\text{no black socks}) = \frac{9}{12}\times\frac{8}{11} = 0.545 $$
Then, we use the complement rule. 
$$P(\text{at least 1 black sock}) = 1-0.545 = \fbox{0.455} $$
\item 0
\item We are interested in the union of three mutually exclusive events.
\begin{align*}
P(\text{2 blues} \OR \text{2 grays} \OR \text{2 blacks}) 
 ~ &=~ P(\text{2 blues}) ~+~ P(\text{2 grays}) ~+~ P(\text{2 blacks}) \\
  &=~ \frac{4}{12}\cdot\frac{3}{11} ~~+~~ \frac{5}{12}\cdot\frac{4}{11} ~~+~~ \frac{3}{12}\cdot\frac{2}{11} \\
  &\approx~ \fbox{0.288}
\end{align*}
\end{enumerate}

\item \begin{enumerate}
\item When drawing without replacement, we can calculate conditional probabilities by considering which chips are left. After a blue is drawn, we have 5 reds, 2 blues, and 2 oranges.
$$P(B_2 | B_1) ~=~ \frac{2}{9} ~=~ 0.22222 $$
\item After an orange is drawn, we have 5 reds, 3 blues, and 1 oranges.
$$P(B_2 | O_1) ~=~ \frac{3}{9} ~=~ 0.3333 $$
\item We use the general rule for joint probabilities.
\begin{align*}
P(B_1 \AND B_2) &= P(B_1)\cdot P(B_2|B_1) \\
&= \frac{3}{10} \cdot \frac{2}{9} \\
&\approx \fbox{0.0666}
\end{align*}
\item Nope. The first events differently change the probabilities of second events.
\end{enumerate}

\newpage

\item \begin{enumerate}
\item To calculate this joint probability, we multiply a marginal and a conditional.
\begin{align*}
P(H_1 \AND PF_2) &= P(H_1) \cdot P(PF_2 | H_1)  \\\\
&= \frac{28}{95} \cdot \frac{59}{94} \\\\
&\approx \fbox{0.185}
\end{align*}

\item This one is a bit more difficult because some fiction books are hardcovers, and we are sampling without replacement.
\begin{align*}
P(F_1 \AND H_2) &= P\Big([HF_1 \AND H_2] \OR [PF_1 \AND H_2] \Big) \\
&= P\big(HF_1 \AND H_2\big) ~~+~~ P\big(PF_1 \AND H_2\big)\\
&= P(HF_1)\cdot P(H_2|HF_1) ~~+~~  P(PF_1)\cdot P(H_2|PF_1)\\\\
&= \frac{13}{95}\cdot \frac{27}{94} ~~+~~ \frac{59}{95} \cdot \frac{28}{94}\\\\
&=  \fbox{0.2243001}
\end{align*}

\item This is easier; we are sampling with replacement, so the draws are independent.
\begin{align*}
P(F_1 \AND H_2) &= P(F) \cdot P(H) \\\\
&= \frac{72}{95} \cdot \frac{28}{95} \\\\
&= \fbox{0.2233795}
\end{align*}

\item The answers to parts (b) and (c) are similar because we are only sampling 2 items from a relatively large population (population size = 95). A rule of thumb is when sample size is less than 10\% of the population, an independence approximation is warranted.
\end{enumerate}

\item First, how many people are wearing leggings?
$$24-7-4-8 =  5$$
We need to consider that there are multiple ways to end up with \#leggings = 1 and \#jeans = 2.
\begin{align*}
P(\text{2 jeans and 1 leggings}) &= P(L_1J_2J_3 \OR J_1L_2J_3 \OR J_1J_2L_3) \\
&= P(L_1J_2J_3) + P(J_1L_2J_3) + P(J_1J_2L_3) \\\\
&= \frac{5\cdot 7 \cdot 6}{24\cdot 23\cdot 22} + \frac{7\cdot 5 \cdot 6}{24\cdot 23\cdot 22} + \frac{7\cdot 6 \cdot 5}{24\cdot 23\cdot 22} \\ \\
&= \fbox{0.05187747}
\end{align*}

\item There are 365 days in a year.
\begin{enumerate}
\item We just need to consider the chance that the second birthday matches the first birthday.
$$P(\text{first two share a birthday}) = \frac{1}{365} \approx  0.002739726$$
\item We recognize ``at least two share'' is the complement of ``nobody shares''.
$$P(\text{nobody shares}) ~=~ 1\times \frac{364}{365} \times \frac{363}{365} ~=~ 0.9917958 $$

$$P(\text{at least 2 share}) ~=~ 1-P(\text{nobody shares}) ~=~ \fbox{0.0082} $$
\end{enumerate}

\item \begin{enumerate}
\item $0.13 \times 100 = \fbox{13}$
\item No. Students at a gym are probably more health conscious than average students.
\end{enumerate}

\item \begin{enumerate}
\item The four events each have a different payoff and probability.
\begin{center}
\renewcommand{\arraystretch}{1.3}
\begin{tabular}{|c|c|c|c|c|} \hline
$i$ & $x_i$ & $P(X=x_i)$ & $x_i \cdot P(X=x_i)$ & $(x_i-\mu)^2 \cdot P(X=x_i)$\\ \hline
1 & \$0 & 0.5 & 0 & 8.55 \\
2 & \$5 & 1/4 & 1.25 & 0.19 \\
3 & \$10 & 12/52 & 2.308 & 7.94\\
4 & \$30 & 1/52 & 0.577 & 12.87\\ \hline
\multicolumn{3}{r|}{Totals:} & $\mu=4.135$ & $\sigma^2=29.54$ \\\cline{4-5}
\end{tabular}
\end{center}
We calculate the standard deviation from the variance.
$$\sigma = \sqrt{29.54} \approx 5.44 $$
The expected winnings are \$4.14 with a standard deviation of \$5.44.
\item I'm rather risk neutral with these small amounts of money. So, I'm willing to pay \$4.14 to play this game. This would give me a \$0 expected payoff (better than a casino!).
\end{enumerate}

\item \begin{enumerate}
\item Let's first calculate the probabilities.
$$P(H_1 \AND H_2 \AND H_3) ~=~ \frac{13\cdot 12 \cdot 11}{52 \cdot 51 \cdot 50} ~=~ 0.01294118 $$
$$P(B_1 \AND B_2 \AND B_3) ~=~ \frac{26\cdot 25 \cdot 24}{52 \cdot 51 \cdot 50} ~=~ 0.1176471 $$

$$P(\text{other}) = 1-0.0129-0.1176 = 0.8695$$

We make a table. Notice the textbook uses rows where I use columns.

\begin{center}
\renewcommand{\arraystretch}{1.3}
\begin{tabular}{|c|c|c|c|c|} \hline
$i$ & $x_i$ & $P(X=x_i)$ & $x_i \cdot P(X=x_i)$ & $(x_i-\mu)^2 \cdot P(X=x_i)$\\ \hline
1 & \$0 & 0.8695 & 0 & 11.18 \\
2 & \$50 & 0.0129 & 0.645  & 27.79 \\
3 & \$25 & 0.1176 & 2.940 & 53.93\\ \hline
\multicolumn{3}{r|}{Totals:} & $\mu=3.585$ & $\sigma^2=92.9$ \\\cline{4-5}
\end{tabular}
\end{center}
We calculate the standard deviation from the variance.
$$\sigma = \sqrt{92.9} \approx 9.64 $$
The expected winnings are \$3.59 with a standard deviation of 9.64. 
\item The expected winnings would be -1.41 USD.
$$3.59-5 = -1.41 $$ 
The variability does not change. Every value under $x_i$ is decreased by 5, but so is the mean, so the last column will be unchanged.
\begin{center}
\renewcommand{\arraystretch}{1.3}
\begin{tabular}{|c|c|c|c|c|} \hline
$i$ & $x_i$ & $P(X=x_i)$ & $x_i \cdot P(X=x_i)$ & $(x_i-\mu)^2 \cdot P(X=x_i)$\\ \hline
1 & -5 & 0.8695 & -4.35 & 11.18 \\
2 & 45 & 0.0129 & 0.58  & 27.79 \\
3 & 20 & 0.1176 & 2.35 & 53.93\\ \hline
\multicolumn{3}{r|}{Totals:} & $\mu=-1.41$ & $\sigma^2=92.9$ \\\cline{4-5}
\end{tabular}
\end{center}
We can also use rules in chapter 2.4 (where I've replaced $a$ and $b$ with 1s).
$$E(X+Y) = E(X) + E(Y) $$
$$Var(X+Y) = Var(X) + Var(Y) $$
The expected value of a constant is the constant. The variance of a constant is 0.
$$E(X-5) = E(X) -5 = 3.59-5 = -1.41$$
$$Var(X-5) = Var(X) + Var(-5) = 92.9-0 = 92.9 $$
\item No. It has a negative expected value.
\end{enumerate}


\end{enumerate}
\end{document}
