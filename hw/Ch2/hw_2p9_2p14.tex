\documentclass[12pt,letterpaper]{article}
\usepackage[utf8]{inputenc}
\usepackage{amsmath}
\usepackage{amsfonts}
\usepackage{amssymb}
\usepackage{amsthm}
\usepackage{graphicx}
\usepackage{tabularx}
\usepackage[left=1cm,right=2cm,top=2cm,bottom=2cm]{geometry}
\usepackage{multicol}
\usepackage{lastpage}
\usepackage{fancyhdr}
\usepackage{multirow,array}
\usepackage{newtxtext,newtxmath}
\usepackage{lastpage}
\usepackage{enumitem}
\newcolumntype{Y}{>{\centering\arraybackslash}X}
\pagestyle{fancy}
\fancyhf{}
\lhead{\textsc{BHCC Mat-181}}
\chead{\textsc{Answers}}
\rhead{\textsc{HW Exercises 2.9-2.14}}
\rfoot{Page \thepage ~of \pageref{LastPage}}
\setenumerate[1]{label={\bf 2.\theenumi: }}
\setenumerate[2]{label={\bf (\theenumii): }}
\setenumerate[3]{label={\bf \theenumiii: }}

\begin{document}
\newcommand{\AND}{\textsc{~and~}}
\newcommand{\OR}{\textsc{~or~}}

\begin{enumerate}
\setcounter{enumi}{8}
\item \begin{enumerate}
\item Independent, not disjoint.
\item Dependent, not disjoint.
\item No. 
\end{enumerate}

\item \begin{enumerate}
\item Let event $X_i$ represent ``Nancy gets the $i$th question right'', where $i$ is an integer such that $1\le i \le 5$. On each question, Nancy has a 25\% chance of success. 
\begin{align*}
P(X_1) &= 0.25 & P(X_2) &= 0.25 & P(X_3) &= 0.25 & P(X_4) &= 0.25 & P(X_5) &= 0.25 \\ 
P(X_1^c) &= 0.75 & P(X_2^c) &= 0.75 & P(X_3^c) &= 0.75 & P(X_4^c) &= 0.75 & P(X_5^c) &= 0.75  
\end{align*}
We are asked to consider the possibility of Nancy missing the first four questions and getting the fifth question.
$$P\left(\begin{matrix}\text{``The first question she}\\ \text{gets right is the 5th question''} \end{matrix}\right) = P\big(X_1^c \AND X_2^c \AND X_3^c \AND X_4^c \AND X_5\big)$$
The five elementary events are independent, so we can find the joint probability by multiplying the marginal probabilities. (Refer to the Multiplication Rule for independent processes on page 86.)
\begin{align*}
P\big(X_1^c \AND X_2^c \AND X_3^c \AND X_4^c \AND X_5\big) &= P(X_1^c) \cdot P(X_2^c)\cdot P(X_3^c)\cdot P(X_4^c)\cdot P(X_5) \\
&= 0.75\cdot 0.75\cdot 0.75\cdot 0.75\cdot 0.25\\
&= 0.75^4\cdot 0.25 \\
&\approx \fbox{0.0791}
\end{align*}
Thus, the probability is about 7.9\%.

\item We hope to determine the probability that Nancy gets all five correct. Again, the elementary events are independent, so we can use the Multiplication Rule.
\begin{align*}
P\big(X_1 \AND X_2 \AND X_3 \AND X_4 \AND X_5\big) &= P(X_1) \cdot P(X_2)\cdot P(X_3)\cdot P(X_4)\cdot P(X_5) \\
&= 0.25\cdot 0.25\cdot 0.25\cdot 0.25\cdot 0.25\\
&= 0.25^5 \\
&\approx \fbox{0.000977}
\end{align*}
Thus, the probability is about 0.098\%.

\item The event ``at least one right'' is the complement of ``all wrong''. It is easy to calculate the probability of ``all wrong''.
\begin{align*} 
P(\text{``all wrong''}) &= P\big(X_1^c \AND X_2^c \AND X_3^c \AND X_4^c \AND X_5^c\big) \\
&= P(X_1^c) \cdot P(X_2^c)\cdot P(X_3^c)\cdot P(X_4^c)\cdot P(X_5^c) \\
&= 0.75^5 \\
&\approx 0.237
\end{align*}
Then we can use the Complement Rule (see page 84), which states, ``For any event $A$ and its complement $A^c$, the probabilities add to 1.''
\begin{align*}
P(A) &= 1 - P(A^c) \\
P(\text{``at least one right''}) &= 1 - P(\text{``all wrong''})\\
&\approx 1 - 0.237\\
&= \fbox{0.763}
\end{align*}
\end{enumerate}

\item \begin{enumerate}
\item $0.16+0.09 = \fbox{0.25}$
\item $0.17+0.09 = \fbox{0.26}$
\item Let event $A$ represent ``random man has at least Bachelor's degree''. Let event $B$ represent ``random woman has at least Bachelor's degree''. Let's assume that the man and woman are each selected randomly, such that the simple events are {\bf independent}. This independence allows us to use the Multiplication Rule for independent processes on page 86. 
\begin{align*}
P(A\AND B) &= P(A)\cdot P(B) \\
&= 0.25 \cdot 0.26 \\
&= \fbox{0.065}
\end{align*}
\item The assumption of independence is not reasonable. Marriage tends to occur between people of similar academic achievement levels.
\end{enumerate}

\item \begin{enumerate}
\item ``Missing 0 days'' is the only other possible outcome. The probabilities of all four outcomes should add to 1.
\begin{align*}
P(\text{``a student misses 0 days''}) &= 1-0.25-0.15-0.28\\
&= \fbox{0.32}
\end{align*}
\item These outcomes are disjoint (mutually exclusive), so we can use the Addition Rule of disjoint outcomes (page 79).
\begin{align*}
P(\text{``student misses no more than 1 day''}) &= P\big(\text{``misses zero days'' \OR ``misses one day''}\big)\\
&= P(\text{``misses zero days''}) + P(\text{``misses one day''})\\
&= 0.32+0.25\\
&= \fbox{0.57}
\end{align*}
\item We can use the Addition Rule of disjoint outcomes.
\begin{align*}
P(\text{``at least 1 day''}) &= P(\text{``1 day''} \OR \text{``2 days''}\OR \text{``at least 3 days''})\\
&= P(\text{``1 day''}) + P(\text{``2 days''})+P(\text{``at least 3 days''})\\
&= 0.25+0.15+0.28\\
&= \fbox{0.68}
\end{align*}
We could have also used the Complement Rule.
\begin{align*}
P(\text{``at least 1 day''}) &= 1 - P(\text{``zero days''})\\
&= 1-0.32\\
&= \fbox{0.68}
\end{align*}

\item Let's assume the absences of each child are independent so we can use the Multiplication Rule for independent processes.
\begin{align*}
P(\text{``2 kids have 0 absences''}) &= P(\text{``1st kid misses 0''}\AND \text{``2nd kid misses 0''})\\
&= P(\text{``1st kid misses 0''}) \cdot P(\text{``2nd kid misses 0''})\\
&= 0.32 \cdot 0.32\\
&= 0.32^2\\
&= \fbox{0.1024}
\end{align*}

\item We continue the assumption that each child's attendance is independent of the other, such that the probability above is correct. Each child has a 0.68 chance of missing some school.
\begin{align*}
P(\text{``both miss some''}) &= P(\text{``1st misses some''} \AND \text{``2nd misses some''})\\
&= P(\text{``1st misses some''}) \cdot P(\text{``2nd misses some''})\\
&= 0.68 \cdot 0.68\\
&= 0.68^2\\
&= \fbox{0.4624}
\end{align*}

\item The assumption of independence is not very reasonable. Siblings often get each other sick. Some parents are more lenient about missing school.
\end{enumerate}

\item \begin{enumerate}
\item Invalid. The probabilities sum to 1.2, which is more than 1.
\item Valid. The outcomes are disjoint. Each probability is between 0 and 1. They sum to 1. 
\item Invalid. The probabilities sum to 0.9, which is less than 1.
\item Invalid. There are negative probabilities.
\item Valid. The outcomes are disjoint. Each probability is between 0 and 1. They sum to 1.
\item Invalid. There are negative probabilities.
\end{enumerate}

\item \begin{enumerate}
\item This is a joint probability. $\frac{459}{20000} = \fbox{0.2295}$
\item This is a disjoint probability. $\frac{4657+2524-459}{20000} = \fbox{0.3361}$

Or, you can add up all the relevant numbers.

$\frac{4198+459+727+854+385+99}{20000} = \frac{6722}{20000} = \fbox{0.3361}$
\end{enumerate}


\end{enumerate}
\end{document}
