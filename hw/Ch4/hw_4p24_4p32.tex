\documentclass[12pt,letterpaper]{article}
\usepackage[utf8]{inputenc}
\usepackage{amsmath}
\usepackage{amsfonts}
\usepackage{amssymb}
\usepackage{amsthm}
\usepackage{graphicx}
\usepackage{tabularx}
\usepackage[left=2cm,right=2cm,top=2cm,bottom=2cm]{geometry}
\usepackage{multicol}
\usepackage{listings}
\lstset{ 
basicstyle = \ttfamily,
showstringspaces=false,
columns=fullflexible,
frame = single,
literate={*}{{\char42}}1
         {-}{{\char45}}1
         {"}{{\fontencoding{T1}\selectfont\textquotedbl}}1
         {'}{{\fontencoding{T1}\selectfont\textquotesingle}}1
}
\usepackage{lastpage}
\usepackage{fancyhdr}
\usepackage{multirow,array}
\usepackage{newtxtext,newtxmath}
\usepackage{lastpage}
\usepackage{enumitem}
\newcolumntype{Y}{>{\centering\arraybackslash}X}
\pagestyle{fancy}
\fancyhf{}
\lhead{\textsc{BHCC Mat-181}}
\chead{\textsc{Answers}}
\rhead{\textsc{HW Exercises 4.24-4.32}}
\rfoot{Page \thepage ~of \pageref{LastPage}}
\setenumerate[1]{label={\bf 4.\theenumi: }}
\setenumerate[2]{label={\bf (\theenumii): }}
\setenumerate[3]{label={\bf \theenumiii: }}

\begin{document}
\newcommand{\N}[2]{\mathcal{N}\big(#1,~#2\big)}
\newcommand{\Geo}[1]{\texttt{Geo}\big(#1\big)}
\newcommand{\B}[2]{\mathcal{B}\big(#1,~#2\big)}
\newcommand{\AND}{\textsc{~and~}}
\newcommand{\OR}{\textsc{~or~}}
\newcommand{\zs}{z^{\star}}

\begin{enumerate}
\setcounter{enumi}{23}

\item \begin{enumerate}
\item Yep. The sample is random. The size of the sample is not larger than 10\% of the population (children in a large city), so the draws are independent. The population data does not seem very skewed, nor do we have reason to suspect it is skewed.
\item Okay, but maybe children of this city are not representative of children of the online article. It should say ``the average age at which gifted children in this city first count to 10''.  Anyway, lets perform the hypothesis test.\\
We state the hypotheses.
$$H_0:~~\mu=32$$
$$H_A:~~\mu<32$$
We find the standard error.
$$SE = \frac{4.31}{\sqrt{36}} = 0.718$$
We find the $z$-score.
$$z = \frac{30.69-32}{0.718} = -1.82 $$
We are doing a left-tail test because the alternative is $<$, so the $p$-value is a left area.
$$p\text{-value } = P(Z<-1.82) = 0.034 $$
The $p$-value is less than our significance level (0.10), so we reject the null. We conclude that gifted children in this city on average count to ten before 32 months.
\item We think that the null population $X\sim \N{32}{4.31}$ only produces means as low as (or lower than) 30.69 about 3.4\% of the time. 
\item We determine $z^\star$ such that $P(|Z|<z^\star) = 0.9$ which implies $P(Z<z^\star) = 0.95$.
$$z^\star = \Phi^{-1}(0.95) = 1.64 $$
We use the confidence interval formula.
$$CI ~~=~~ \bar{x} \pm z^\star SE  ~~=~~ 30.69 \pm (1.64)(0.718)$$
$$CI = (29.5,\,31.9)$$
\item Yes, because our confidence interval does not include 32. Both strategies suggest the sample did not come from a population with $\mu=32$ and $\sigma=4.31$.
\end{enumerate}

\item \begin{enumerate}
\item We need random sampling, independence, sample size over 30, and the population to not have a strong skew. The sample size is surely less than 10\% of all ER visits. I don't have any reason to think wait times would have a strong skew, so we will assume the skew is not too strong.
\item We state the hypotheses.
$$H_0: ~~ \mu = 127 $$
$$H_A: ~~ \mu \ne 127 $$
We calculate the standard error.
$$SE ~~=~~ \frac{39}{\sqrt{64}} ~~=~~ 4.875 $$
We calculate a $z$-score.
$$z = \frac{137.5-127}{4.875} = 2.15 $$
We calculate the two-tailed $p$-value.
$$p\text{-value} ~~=~~ P(|Z|<2.15) ~~=~~ 2P(Z<-2.15) ~~=~~ 2\Phi(-2.15) ~~=~~ 0.0316 $$
We compare the $p$-value to $\alpha$.
$$0.0316 < 0.05 $$
Because $p$-value $< \alpha$, we reject the null hypothesis. We conclude that the average wait time has increased.
\item The conclusion would change: we would retain the null. This is because 0.0316 is larger than 0.01.
\end{enumerate}

\item \begin{enumerate}
\item We state the hypotheses.
$$H_0:~~\mu = 100$$
$$H_A:~~\mu \ne 100$$
We calculate the standard error.
$$SE = \frac{6.5}{\sqrt{36}} = 1.083 $$
We calculate a $z$ score.
$$z ~~=~~ \frac{118.2-100}{6.5} ~~=~~ 2.8 $$
We determine the two-tail probability.
$$p\text{-value} ~~=~~ P(|Z|<2.8) ~~=~~ 2P(Z<-2.8) ~~=~~ 2\Phi(-2.8) ~~=~~ 0.00511 $$
This is smaller than 0.1. We reject the null hypothesis. We conclude that the average IQ of mothers of gifted children is higher than 100.
\item We already have $SE=1.083$. We need $z^\star$ such that $P(|Z|<z^\star) = 0.90$. If you draw a sketch of a normal curve, you'll notice this $z^\star$ also satisfies $P(Z<z^\star) = 0.95$.
$$z^\star = \Phi^{-1}(0.95) = 1.64 $$
We find the confidence interval.
$$CI ~~=~~ 118.2 \pm (1.64)(1.083) $$
$$CI ~~=~~ (116.4,\,120.0) $$
\item Yes. Both strategies suggest that these mothers' average IQ is not 100.
\end{enumerate}

\item We want to find $\bar{x}$ such that $P\left(Z>\frac{\bar{x}-30}{10/\sqrt{70}}\right)=0.05$. Let $z_{\alpha} = \frac{\bar{x}-30}{10/\sqrt{70}}$. We want to find $z_{\alpha}$ such that $P\left(Z>z_{\alpha}\right)=0.05$. If you draw a sketch, or use the reasoning of a complement, then it should be clear that we want a left area of 0.95.
$$P(Z<z_{\alpha}) = 0.95 $$
We determine $z_{\alpha}$.
$$z_{\alpha} ~=~ \Phi^{-1}(0.95) ~=~ 1.645 $$
So,
$$1.645 = \frac{\bar{x}-30}{10/\sqrt{70}} $$
We solve for $\bar{x}$.
$$\bar{x} ~~=~~ \frac{1.645 \times 10}{\sqrt{70}} + 30 ~~=~~ 31.97$$

\item We want to find $\bar{x}$ such that $P\left(|Z|>\frac{\bar{x}-30}{10/\sqrt{70}}\right)=0.05$. From the $p$-value of 0.05, we can determine a two-tail $z$ score.
$$P(|Z|>z) = 0.05 $$
$$P(Z<-z) = 0.025 $$
$$z ~~=~~ -\Phi^{-1}(0.025) ~~=~~ 1.96$$
Thus,
$$1.96 = \frac{\bar{x}-30}{10/\sqrt{70}} $$
Do some algebra.
$$\bar{x} = 32.34 $$


\item \begin{enumerate}
\item Diana's skeptical position is that any changes in her symptoms are due to chance, not due to the drug.
\item The Type 1 Error is deciding the drug helps when it doesn't.
\item The Type 2 Error is deciding the drug does not help when it does.
\end{enumerate}


\item \begin{enumerate}
\item The null hypothesis claims the restaurant is sanitary. The alternative hypothesis claims the restaurant is unsanitary.
\item The Type 1 Error would be deciding the restaurant is unsanitary when it is actually sanitary.
\item The Type 2 Error would be deciding the restaurant is sanitary when it is actually unsanitary.
\item Type 1 Error is more problematic for the restaurant owner because it is like a false conviction. The restaurant's license would be revoked when it shouldn't have been.
\item Type 2 Error is more problematic for the diners. They can find another restaurant if this one closes, but if it stays open while unsanitary, they might get sick.
\item I'd prefer if the safety inspector only required strong evidence. I don't like getting food poisoning. Sure, some restaurants will be closed when they shouldn't be, but there will be fewer unsanitary restaurants.
\end{enumerate}

\item \begin{enumerate}
\item The standard error of a sample mean is larger with a smaller sample size. {\bf Scenario I} has a larger standard error.
\item A higher confidence level has a higher margin of error. {\bf Scenario I} has a larger margin of error.
\item Same. The sample size does not affect the calculation of the $p$-value for a given $z$-score.
\item We are more likely to make a Type 2 Error when we have a stricter cutoff. {\bf Scenario I} has a larger chance of Type 2 Error.
\end{enumerate} 

\item \begin{enumerate}
\item True. A larger confidence level gives a wider confidence interval.
\item False. Increasing the significance level will increase the probability of making a Type 1 Error.
\item False. We didn't reject the possibility that $\mu=5$. With more data we might. We could say, ``Under this scenario, the true population mean {\bf might be} 5.''
\item True. The power of the test is the chance of NOT making Type 2 Error when the alternative is true.
\item True. For small effect sizes, we require very large sample sizes; however, these small effect sizes are usually not very interesting...
\end{enumerate}




\end{enumerate}
\end{document}
