\documentclass[12pt,letterpaper]{article}
\usepackage[utf8]{inputenc}
\usepackage{amsmath}
\usepackage{amsfonts}
\usepackage{amssymb}
\usepackage{amsthm}
\usepackage{graphicx}
\usepackage{tabularx}
\usepackage[left=2cm,right=2cm,top=2cm,bottom=2cm]{geometry}
\usepackage{multicol}
\usepackage{listings}
\lstset{ 
basicstyle = \ttfamily,
showstringspaces=false,
columns=fullflexible,
frame = single,
literate={*}{{\char42}}1
         {-}{{\char45}}1
         {"}{{\fontencoding{T1}\selectfont\textquotedbl}}1
         {'}{{\fontencoding{T1}\selectfont\textquotesingle}}1
}
\usepackage{lastpage}
\usepackage{fancyhdr}
\usepackage{multirow,array}
\usepackage{newtxtext,newtxmath}
\usepackage{lastpage}
\usepackage{enumitem}
\newcolumntype{Y}{>{\centering\arraybackslash}X}
\pagestyle{fancy}
\fancyhf{}
\lhead{\textsc{BHCC Mat-181}}
\chead{\textsc{Answers}}
\rhead{\textsc{HW Exercises 4.7-4.16}}
\rfoot{Page \thepage ~of \pageref{LastPage}}
\setenumerate[1]{label={\bf 4.\theenumi: }}
\setenumerate[2]{label={\bf (\theenumii): }}
\setenumerate[3]{label={\bf \theenumiii: }}

\begin{document}
\newcommand{\N}[2]{\mathcal{N}\big(#1,~#2\big)}
\newcommand{\Geo}[1]{\texttt{Geo}\big(#1\big)}
\newcommand{\B}[2]{\mathcal{B}\big(#1,~#2\big)}
\newcommand{\AND}{\textsc{~and~}}
\newcommand{\OR}{\textsc{~or~}}
\newcommand{\zs}{z^{\star}}

\begin{enumerate}
\setcounter{enumi}{16}

\item \begin{enumerate}
\item The null hypothesis claims New Yorkers sleep 8 hours on average. 
$$H_0: ~~\mu = 8 $$
The alternative hypothesis claims New Yorkers sleep less than 8 hours on average.
$$H_A: ~~\mu < 8 $$
\item The null hypothesis claims employees waste 15 minutes on average.
$$H_0: ~~\mu=15 $$
The alternative hypothesis claims employees waste more than 15 minutes on average.
$$H_A: ~~\mu > 15 $$
\end{enumerate}

\item \begin{enumerate}
\item The null hypothesis claims the average calories is 1100.
$$H_0: ~~\mu = 1100 $$
The alternative hypothesis claims the average calories is not 1100.
$$H_A: ~~\mu \ne 1100 $$
\item The null hypothesis claims the population's average score is 462.
$$H_0: ~~\mu=462 $$
The alternative hypothesis claims the population's average score is not 462.
$$H_A: ~~\mu \ne 462 $$
\end{enumerate}

\item The claims should be about the population mean, not the sample mean. The null should be an equality. Both hypotheses should involve the same number (10).
$$H_0:~~\mu = 10 $$
$$H_A:~~\mu > 10 $$

\item The claims should be about the population parameter. The alternative should be an inequality because she is interested in a value being higher or lower.
$$H_0:~~\mu = 23.44 $$
$$H_A:~~\mu \ne 23.44 $$

\item \begin{enumerate}
\item That claim is not supported. Our confidence interval has a maximum of 2.45 hours.
\item Sure. Our confidence interval is (2.133, 2.45) hours.
\item Yep. A 99\% confidence interval will be even larger, so it will still straddle $2.2$.
\end{enumerate}


%%%%%%%%%%%%%%%%% 4.22
\item \begin{enumerate}
\item I think her claim is outside of the confidence interval, so I am skeptical. Maybe she is just estimating to the nearest power of 10? 
\item Nope. A 90\% confidence interval is even smaller! 
\end{enumerate}


%%%%%%%%%%%%%%%%% 4.23
\item We will use a 5\% significance level. We state the hypotheses.
$$H_0: ~~ \mu = 130 $$
$$H_A: ~~ \mu \ne 130  $$
We assume $\sigma \approx 17$ and calculate the standard error.
$$SE = \frac{17}{\sqrt{35}} = 2.87$$
We find a $z$-score.
$$z = \frac{\bar{x}-\mu}{SE} = \frac{134-130}{2.87} = 1.39 $$
Because the alternative hypthesis is $\ne$, we find a two-tail area.
$$p\text{-value} ~=~ P(|Z|>1.39) ~=~ 2P(Z<-1.39) ~=~ 0.16 $$
The $p$-value is bigger than 5\%, we retain the null. We do not have sufficient evidence to claim the bags are lying.

\end{enumerate}
\end{document}
