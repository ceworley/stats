\documentclass[12pt,letterpaper]{article}
\usepackage[utf8]{inputenc}
\usepackage{amsmath}
\usepackage{amsfonts}
\usepackage{amssymb}
\usepackage{amsthm}
\usepackage{graphicx}
\usepackage{tabularx}
\usepackage[left=2cm,right=2cm,top=2cm,bottom=2cm]{geometry}
\usepackage{multicol}
\usepackage{listings}
\lstset{ 
basicstyle = \ttfamily,
showstringspaces=false,
columns=fullflexible,
frame = single,
literate={*}{{\char42}}1
         {-}{{\char45}}1
         {"}{{\fontencoding{T1}\selectfont\textquotedbl}}1
         {'}{{\fontencoding{T1}\selectfont\textquotesingle}}1
}
\usepackage{lastpage}
\usepackage{fancyhdr}
\usepackage{multirow,array}
\usepackage{newtxtext,newtxmath}
\usepackage{lastpage}
\usepackage{enumitem}
\newcolumntype{Y}{>{\centering\arraybackslash}X}
\pagestyle{fancy}
\fancyhf{}
\lhead{\textsc{BHCC Mat-181}}
\chead{\textsc{Answers}}
\rhead{\textsc{HW Exercises 4.7-4.16}}
\rfoot{Page \thepage ~of \pageref{LastPage}}
\setenumerate[1]{label={\bf 4.\theenumi: }}
\setenumerate[2]{label={\bf (\theenumii): }}
\setenumerate[3]{label={\bf \theenumiii: }}

\begin{document}
\newcommand{\N}[2]{\mathcal{N}\big(#1,~#2\big)}
\newcommand{\Geo}[1]{\texttt{Geo}\big(#1\big)}
\newcommand{\B}[2]{\mathcal{B}\big(#1,~#2\big)}
\newcommand{\AND}{\textsc{~and~}}
\newcommand{\OR}{\textsc{~or~}}
\newcommand{\zs}{z^{\star}}

\begin{enumerate}
\setcounter{enumi}{6}
%%%%%%%%%%%%%% 4.7
\item $CI = 0.45 \pm (1.96)(0.012) = (0.426,\,0.474)$\\
We are 95\% confident the true population proportion is between 42.6\% and 47.4\%.

%%%%%%%%%%%%% 4.8
\item We need to determine $z^{\star}$ such that $P(|Z| \le z^{\star}) = 0.99$. We draw a sketch of a centrally symmetric area of 0.99, leaving two tails, each with 0.005 area. We can use the $z$ table in reverse to find $\zs$ such that $P(Z<\zs) = 0.995$.  We determine that $\zs = 2.58$. 
$$CI = 0.52 \pm (2.58)(0.024) = (0.458,\,0.582)$$
We are 99\% confident that the true population proportion is between 0.458 and 0.582.

%%%%%%%%%%%% 4.9
\item \begin{enumerate}
\item False, we only have some level of confidence.
\item True, this is was a condidence interval is.
\item True, the entire confidence interval is below 50\%.
\item False. This is not what standard error is. Standard error is the standard deviation of a sampling distribution. The standard error comes from the random samples being different, not from the individuals.
\end{enumerate}


%%%%%%%%%%%% 4.10
\item \begin{enumerate}
\item False. The confidence interval straddles the 50\% mark.
\item False. If the poll reached 97.6\% of users, the standard error would be tiny (of course we are not sure how to deal with this situation exactly because it would mean sampling more than 10\% of the population). Standard error comes from differences between random samples. It does decrease with larger sample sizes, but it is often not even a percentage... 
\item False. A higher sample size gives a smaller standard error.
\item False. A higher confidence level has a wider confidence interval.
\end{enumerate}

%%%%%%%%%%%% 4.11
\item \begin{enumerate}
\item We are 95\% confident that the true population mean is between 1.38 and 1.92 hours.
\item The confidence level is higher.
\item Larger sample size leads to smaller margin of error.
\end{enumerate}

\item \begin{enumerate}
\item .
\end{enumerate}

\item \begin{enumerate}
\item .
\end{enumerate}


\end{enumerate}
\end{document}
