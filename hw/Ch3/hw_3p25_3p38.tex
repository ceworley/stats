\documentclass[12pt,letterpaper]{article}
\usepackage[utf8]{inputenc}
\usepackage{amsmath}
\usepackage{amsfonts}
\usepackage{amssymb}
\usepackage{amsthm}
\usepackage{graphicx}
\usepackage{tabularx}
\usepackage[left=2cm,right=2cm,top=2cm,bottom=2cm]{geometry}
\usepackage{multicol}
\usepackage{listings}
\lstset{ 
basicstyle = \ttfamily,
showstringspaces=false,
columns=fullflexible,
literate={*}{{\char42}}1
         {-}{{\char45}}1
         {"}{{\fontencoding{T1}\selectfont\textquotedbl}}1
         {'}{{\fontencoding{T1}\selectfont\textquotesingle}}1
}
\usepackage{lastpage}
\usepackage{fancyhdr}
\usepackage{multirow,array}
\usepackage{newtxtext,newtxmath}
\usepackage{lastpage}
\usepackage{enumitem}
\newcolumntype{Y}{>{\centering\arraybackslash}X}
\pagestyle{fancy}
\fancyhf{}
\lhead{\textsc{BHCC Mat-181}}
\chead{\textsc{Answers}}
\rhead{\textsc{HW Exercises 3.25-3.38}}
\rfoot{Page \thepage ~of \pageref{LastPage}}
\setenumerate[1]{label={\bf 3.\theenumi: }}
\setenumerate[2]{label={\bf (\theenumii): }}
\setenumerate[3]{label={\bf \theenumiii: }}

\begin{document}
\newcommand{\N}[2]{\mathcal{N}\big(#1,~#2\big)}
\newcommand{\Geo}[1]{\texttt{Geo}\big(#1\big)}
\newcommand{\B}[2]{\mathcal{B}\big(#1,~#2\big)}
\newcommand{\AND}{\textsc{~and~}}
\newcommand{\OR}{\textsc{~or~}}

We will use the notation $X\sim \B{n}{p}$ to say $X$ is binomially distributed with $n$ trials and chance of success $p$ (on each trial). 

\begin{enumerate}
\setcounter{enumi}{24}
\item \begin{enumerate}
\item Yes. Each trial is independent and has the same probability of success (and each trial will either be a success or failure).
\item Let $X\sim \B{10}{0.697}$. We are asked to find the probability $P(X=6)$.
\begin{align*}
P(X=6) ~~&=~~ {10\choose 6} (0.697)^6(1-0.697)^4 \\
&=~~ 210 \times 0.697^6 \times 0.303^4 \\
&=~~ \fbox{0.203}
\end{align*}
\item This is the same thing as part (b), just worded differently. Each child either did or didn't. If 6 did, then 4 didn't.
$$\fbox{0.203} $$
\item Let $X\sim \B{{n=\bf 5}}{p=0.697}$. We can calculate the entire probability mass function.
\begin{center}
\renewcommand{\arraystretch}{1.5}
\begin{tabular}{|c|c|c|}\hline
$k$ & $P(X=k)$ before simplification & $P(X=k)$ \\ \hline
0  & ${5\choose 0} 0.697^0 \times0.303^5$ & 0.00255395 \\
1  & ${5\choose 1} 0.697^0 \times 0.303^5 $ & 0.02937469 \\
2  & ${5\choose 2} 0.697^2 \times 0.303^3$ & 0.13514297 \\
3  & ${5\choose 3} 0.697^3 \times 0.303^2$ & 0.31087342 \\
4  & ${5\choose 4} 0.697^4 \times 0.303^1$ & 0.35755573 \\
5  & ${5\choose 5} 0.697^5 \times 0.303^0$ & 0.16449924 \\ \hline
\end{tabular}
\end{center}
We are asked to determine the probability $P(X\le 2)$.
\begin{align*}
P(X\le 2) ~~&=~~ P(X=0)+P(X=1)+P(X=2) \\
 &=~~ 0.00255 + 0.02937 + 0.13514 \\
 &\approx~~ \fbox{0.167}
\end{align*}
Also, if you are interested... R will evaluate this with a single command: \begin{center}
\texttt{pbinom(2, 5, 0.697)}
\end{center}
Also, we can use \texttt{dbinom(k,n,p)} to get the simple probabilities.
\begin{verbatim}
> dbinom(seq(0,5), 5, 0.697)
[1] 0.00255395 0.02937469 0.13514297 0.31087342 0.35755573 0.16449924
> pbinom(2, 5, 0.697)
[1] 0.1670716
\end{verbatim}

\item (Same distribution as (d).) It is easiest to recognize the complement of ``at least 1'' is ``none''.
\begin{align*}
P(X\ge 1)  ~~&=~~ 1-P(X=0) \\
  &=~~1-0.00255 \\
  &=~~\fbox{0.99745}
\end{align*}
\end{enumerate}

\newpage

\item \begin{enumerate}
\item Yes. Each trial will be ``success'' or ``fail'', and each trial has the same probability of ``success''.
\item Let $X\sim \B{100}{0.90}$. We are asked to calculate the probability $P(X=97)$.
\begin{align*}
P(X = 97) ~~&=~~ {100\choose 97}(0.9)^{97}(0.1)^3\\
            &=~~ 161700\times 0.9^{97}\times0.1^3\\
            &=~~ \fbox{0.005891602}
\end{align*}
We could also use R.
\begin{verbatim}
> dbinom(97,100,0.9)
[1] 0.005891602
\end{verbatim}
\item This is the same question as (b).
$$\fbox{0.00589} $$
\item Let $Y\sim \B{10}{0.9}$, where $Y$ is the number of people who had chickenpox from a random sample of 10 Americans. We recognize ``at least one'' as the complement of ``none''.
\begin{align*}
P(Y = 0) ~~&=~~ {10\choose 0}(0.9)^{0}(0.1)^{10}\\
            &=~~ 0.000000001
\end{align*}
We use the complement rule.
\begin{align*}
P(Y \ge 1) ~~&=~~ 1 - P(Y = 0)\\
            &=~~ 1-0.000000001\\
            &=~~ 0.999999999\\
            &\approx~~ \fbox{1}
\end{align*}
\item We continue to let $Y\sim \B{10}{0.9}$. We rephrase the question in terms of number of people who \emph{had} chickenpox. We want to find the probability that \emph{at least} 7 Americans had chickenpox.
\begin{align*}
P(Y \ge 7) ~~&=~~ P(Y=7)+P(Y=8)+P(Y=9)+P(Y=10) \\
 &=~~ {10\choose 7}(0.9)^{7}(0.1)^{3}+
      {10\choose 8}(0.9)^{8}(0.1)^{2}+
      {10\choose 9}(0.9)^{9}(0.1)^{1}+
     {10\choose 10}(0.9)^{10}(0.1)^{0}\\
 &\approx~~ \fbox{0.9872}
\end{align*}
\end{enumerate}
\newpage

\item Let $X\sim \B{50}{0.70}$. As a reminder, that means $n=50$ and $p=0.7$.
\begin{enumerate}
\item We can calculate the mean and standard deviation of $X$ using the formulas for binomial distributions.
$$\mu = np = 50\times 0.70 = \fbox{35} $$
$$\sigma = \sqrt{np(1-p)} = \sqrt{(50)(0.7)(0.3)} = \fbox{3.24} $$
\item Well, 45 would have a pretty high $z$ score. Remember, we can often approximate a binomial distribution with a normal distribution.
 $$z = \frac{45-35}{3.24} = 3.09 $$
Yeah. If $|z|>2$, we consider it unusual, as it happens less than 5\% of the time due to chance. If $|z|>3$, we consider it rare, as it happens less than 0.3\% of the time due to chance.
\item Hmm... we can do this a few ways. The first is to use R.
\begin{verbatim}
> 1 - pbinom(44, 50, 0.7)
[1] 0.0007228617
> sum(dbinom(seq(45,50), 50, 0.7))
[1] 0.0007228617
\end{verbatim}
We can do this by hand...
\begin{center}
\renewcommand{\arraystretch}{1.5}
\begin{tabular}{|c|c|c|}\hline
$k$ & $P(X=k)$ before simplification & $P(X=k)$ \\ \hline
\vdots & \vdots & \vdots \\
45  & ${50\choose 45} (0.7)^{45} (0.3)^5$ &  0.000551 \\
46  & ${50\choose 46} (0.7)^{46} (0.3)^4 $ & 0.000140 \\
47  & ${50\choose 47} (0.7)^{47} (0.3)^3$ &  0.000028 \\
48  & ${50\choose 48} (0.7)^{48} (0.3)^2$ &  0.000004 \\
49  & ${50\choose 49} (0.7)^{49} (0.3)^1$ &  0.000000 \\
50  & ${50\choose 50} (0.7)^{50} (0.3)^0$ &  0.000000 \\ \hline
\end{tabular}
\end{center}
We sum the probabilities to get $P(X\ge 45) = 0.00072$.

As a third technique, we can use the {\bf normal approximation} with {\bf continuity correction}. \\
Let $Y\sim \N{35}{3.24}$.
\begin{align*}
P(X \ge 45) ~~&\approx~~ P(Y \ge 44.5)\\
 &=~~ 1-\Phi\left(\frac{44.5-35}{3.24} \right)\\
 &=~~0.00168
\end{align*}
But, notice the normal approximation did not work out great... it was off by a factor of more than 2. {\bf The normal approximation does not work well in the tails}. However, all three techniques agree that the probability is quite low.
\end{enumerate}

\newpage

\item \begin{enumerate}
\item Let $X\sim \B{120}{0.9}$. To determine the mean and standard deviation, we use the formulas for binomial distributions.
$$\mu = np = (120)(0.9) = 108 $$
$$\sigma = \sqrt{np(1-p)} = \sqrt{(120)(0.9)(0.1)} = 3.286 $$
We expect about 108 with a standard deviation of 3.28 people to have had chickenpox.
\item 105 is within 1 standard deviation of 108.
$$z = \frac{105-108}{3.28} = -0.914$$
So, this does not seem unusual.
\item We are asked to calculate $P(X\le 105)$. This is easy with a computer.
\begin{verbatim}
> pbinom(105, 120, 0.9)
[1] 0.2181634
\end{verbatim}
To do it by hand, we will use a normal approximation with the continuity correction. Let $Y\sim\N{108}{3.286}$.
\begin{align*}
P(X \le 105) ~~&\approx~~ P(Y \le 105.5)\\
 &=~~ \Phi\left(\frac{105.5-108}{3.286} \right)\\
 &=~~ 0.2233872
\end{align*}
So, either way we get about 22\% chance, which agrees with us not being surprised in part (b).
\end{enumerate}




\item Let $X\sim \B{2500}{0.7}$, where $X$ represents the number of students who accept. We will definitely use a normal approximation here. In fact, we are dealing with such a large $n$ we don't even need to make a continuity correction.

We first calculate the mean and standard deviation using the formulas for binomial distributions.
$$\mu = np = 2500 \times 0.7 = 1750 $$
$$\sigma = \sqrt{np(1-p)} = \sqrt{2500(0.7)(0.3)} = 22.9$$
Now, let $Y\sim \N{1750}{22.9}$. The normal approximation (with continuity correction) can be used.
\begin{align*}
P(X > 1786) ~~&\approx~~ P(Y > 1786.5)\\
 &=~~ 1-\Phi\left(\frac{1786.5-1750}{22.9} \right)\\
 &=~~ 0.056
\end{align*}
Let's use R to calculate it more exactly.
\begin{verbatim}
> 1-pbinom(1786,2500,0.7)
[1] 0.05506358
\end{verbatim}
Wow, in this case the normal approximation did a great job!


\item Let $X\sim \B{15000}{0.09}$. Remember, that means $n=15000$ and $p=0.09$. We want to calculate the probability $P(X \ge 1500)$. Using a computer, we can do that easily.
\begin{verbatim}
> 1-pbinom(1499,15000,0.09)
[1] 1.326331e-05
\end{verbatim}
Let's also use a normal approximation. We first calculate the mean and standard deviation.
$$\mu = np = (15000)(0.09) = 1350 $$
$$\sigma = \sqrt{np(1-p)} = \sqrt{(15000)(0.09)(0.91)} = 35.04996 $$
Now we define a continuous random variable $Y\sim \N{1350}{35}$, and we use the normal approximation.
\begin{align*}
P(X \ge 1500) ~~&\approx~~ P(Y > 1499.5)\\
 &=~~ 1-\Phi\left(\frac{1499.5-1350}{35} \right)\\
 &=~~ 0.0000097
\end{align*}
We don't expect the normal approximation to work in the tails. But we know the chance is low!



\item In all cases, the sides are all equally likely, so we can just say 1 in 4 sides corresponds to ``success'', but which side is ``success'' depends on the case. So, we can just define random variable $X\sim\B{3}{0.25}$, where $X$ is the number of successes. We make a table, where we use $k$ to represent specific (possible) outcomes.
\begin{center}
\renewcommand{\arraystretch}{1.5}
\begin{tabular}{|c|c|c|}\hline
$k$ & $P(X=k)$ before simplification & $P(X=k)$ \\ \hline
0  & ${3\choose 0} (0.25)^{0} (0.75)^3$ &  0.421875 \\
1  & ${3\choose 1} (0.25)^{1} (0.75)^2$ & 0.421875 \\
2  & ${3\choose 2} (0.25)^{2} (0.75)^1$ &  0.140625 \\
3  & ${3\choose 3} (0.25)^{3} (0.75)^0$ &  0.015625 \\ \hline
\end{tabular}
\end{center}
\begin{enumerate}
\item Reread as ``at least one success''. This is the complement of ``no successes''.
$$P(X\ge 1) ~~=~~ 1-P(X=0) ~~=~~ 1-0.422 ~~=~~ \fbox{0.578}$$
\item Reread as ``exactly two successes'', and look at the table.
$$P(X = 2) ~~=~~ \fbox{0.141}$$
\item Reread as ``exactly one success'', and look at the table.
$$P(X = 1) ~~=~~ \fbox{0.422}$$
\item Reread as ``at most two success''. This is the complement of ``exactly three successes''.
$$P(X \le 2) ~~=~~ 1 - P(X=3) ~~=~~ 1-0.0156 ~~=~~ \fbox{0.9844} $$

\end{enumerate}


\item Let $X\sim\N{10}{0.07}$, where $X$ represents the number of teenagers suffering from arachnophobia.
\begin{enumerate}
\item We want to calculate $P(X\ge 1)$. This is the complement of $P(X=0)$.
\begin{align*}
P(X\ge 1) ~~&=~~1 - P(X=0)\\
 &=~~ 1-{10\choose 0}(0.07)^0(0.93)^{10}\\
 &=~~ \fbox{0.516}
\end{align*}
\item We want to calculate $P(X = 2)$. 
\begin{align*}
P(X = 2) ~~&=~~ {10\choose 2}(0.07)^2(0.93)^{8}\\
 &=~~ \fbox{0.123}
\end{align*}
\item We want to calculate $P(X \le 1)$. 
\begin{align*}
P(X \le 1) ~~&=~~ {10\choose 0}(0.07)^0(0.93)^{10}+{10\choose 1}(0.07)^1(0.93)^{9}\\
 &=~~ \fbox{0.848}
\end{align*}
\item No. There is a 15\% chance that, in the tent, more than 1 teenager is afraid of spiders.
\end{enumerate}


\item \begin{enumerate}
\item $0.125 \times (1-0.125) = \fbox{0.109}$
\item ${2\choose 1}(0.125)^1(1-0.125)^1 = \fbox{0.219} $
\item ${6\choose 2}(0.125)^2(1-0.125)^4 = \fbox{0.137} $
\item Complement. $1-{6\choose 0}(0.125)^0(1-0.125)^6 = \fbox{0.551} $
\item Geometric. $(1-0.125)^3(0.125) = \fbox{0.0837}$
\item We can calculate a $z$ score. First we need $\mu$ and $\sigma$ of the binomial distribution $\B{6}{0.75}$.
$$\mu = (6)(0.75) = 4.5$$
$$\sigma = \sqrt{(6)(0.75)(0.25)} = 1.06 $$
$$z = \frac{2-4.5}{1.06} = -2.36 $$
This $z$ score is considered unusual because $|-2.36| > 2$.

We could also calculate the probability of having 2 {\bf or fewer} children with brown eyes.
\begin{align*}
P(X\le 2) &= {6\choose 0}(0.75)^0(0.25)^{6} +  {6\choose 1}(0.75)^1(0.25)^{5} + {6\choose 2}(0.75)^2(0.25)^{4}\\
&= 0.0376
\end{align*}
So, having 2 {\bf or fewer} kids with brown eyes only happens about 4\% of the time. This is low enough to be unusual.
\end{enumerate}




\end{enumerate}
\end{document}
