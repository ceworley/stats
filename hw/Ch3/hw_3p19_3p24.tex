\documentclass[12pt,letterpaper]{article}
\usepackage[utf8]{inputenc}
\usepackage{amsmath}
\usepackage{amsfonts}
\usepackage{amssymb}
\usepackage{amsthm}
\usepackage{graphicx}
\usepackage{tabularx}
\usepackage[left=2cm,right=2cm,top=2cm,bottom=2cm]{geometry}
\usepackage{multicol}
\usepackage{listings}
\lstset{ 
basicstyle = \ttfamily,
showstringspaces=false,
columns=fullflexible,
literate={*}{{\char42}}1
         {-}{{\char45}}1
         {"}{{\fontencoding{T1}\selectfont\textquotedbl}}1
         {'}{{\fontencoding{T1}\selectfont\textquotesingle}}1
}
\usepackage{lastpage}
\usepackage{fancyhdr}
\usepackage{multirow,array}
\usepackage{newtxtext,newtxmath}
\usepackage{lastpage}
\usepackage{enumitem}
\newcolumntype{Y}{>{\centering\arraybackslash}X}
\pagestyle{fancy}
\fancyhf{}
\lhead{\textsc{BHCC Mat-181}}
\chead{\textsc{Answers}}
\rhead{\textsc{HW Exercises 3.19-3.24}}
\rfoot{Page \thepage ~of \pageref{LastPage}}
\setenumerate[1]{label={\bf 3.\theenumi: }}
\setenumerate[2]{label={\bf (\theenumii): }}
\setenumerate[3]{label={\bf \theenumiii: }}

\begin{document}
\newcommand{\N}[2]{\mathcal{N}\big(#1,~#2\big)}
\newcommand{\Geo}[1]{\texttt{Geo}\big(#1\big)}
\newcommand{\AND}{\textsc{~and~}}
\newcommand{\OR}{\textsc{~or~}}

\begin{enumerate}
\setcounter{enumi}{18}
\item \begin{enumerate}
\item Nope. There are way more than 2 possible hands in poker. 
\item There are more than 2 possible outcomes when rolling a die. Of course, you could define two mutually exclusive and exhaustive events...
\end{enumerate}

\item \begin{enumerate}
\item With replacement:
$$P(\text{1st female and 2nd female w/ replacement})=\frac{5}{10} \times \frac{5}{10} = \frac{1}{4} $$
Without replacement:
$$P(\text{1st female and 2nd female w/out replacement})=\frac{5}{10} \times \frac{4}{9} = \frac{20}{90} = \frac{2}{9} \approx 0.222 $$
\item With replacement:
$$P(\text{1st female and 2nd female w/ replacement})=\frac{5000}{10000} \times \frac{5000}{10000} = \frac{1}{4} $$
Without replacement:
$$P(\text{1st female and 2nd female w/out replacement})=\frac{5000}{10000} \times \frac{4999}{9999} = 0.249975 \approx 0.25 $$
\item This assumption is reasonable. We see that with a large population (3.20.b) there is a tiny error from the independence approximation.
\end{enumerate}

\item \begin{enumerate}
\item If we assume independence, this is a geometric random variable. Let $X\sim \Geo{p=0.471}$.
$$P\big(X=3\big) ~=~ (1-0.471)^2(0.471) ~=~ 0.1318 $$
\item We assume independence. This is not geometric. We have not named repeated failures, but it is a special version of binomial distribution.
\begin{align*}
P(\text{1st AND 2nd AND 3rd}) &= P(\text{1st}) \cdot P(\text{2nd}) \cdot P(\text{3rd}) \\
&= 0.471^3 \\
&= 0.1044871
\end{align*}
\item We are back to geometric! Let $X\sim \Geo{p=0.471}$. We learned a formula for $\mu$ and $\sigma$ for geometric distributions.
\begin{align*}
\mu &= \frac{1}{p} \\\\
 &= \frac{1}{0.471} \\\\
 &= 2.123142
\end{align*}
\begin{align*}
\sigma &= \frac{\sqrt{1-p}}{p} \\\\
 &= \frac{\sqrt{1-0.471}}{0.471} \\ \\
 &= 1.544212
\end{align*}
On average, we expect sampling 2.12 women before finding a married woman, give or take 1.5 women.

\item We just change $p$.
\begin{align*}
\mu &= \frac{1}{p} \\\\
 &= \frac{1}{0.3} \\\\
 &= 3.33333
\end{align*}
\begin{align*}
\sigma &= \frac{\sqrt{1-p}}{p} \\\\
 &= \frac{\sqrt{1-0.3}}{0.3} \\ \\
 &= 2.788867
\end{align*}
On average, we expect sampling 3.33 women before finding a married woman, give or take 2.79 women.
\item Decreasing $p$ increases both $\mu$ and $\sigma$.
\end{enumerate}

\item \begin{enumerate}
\item This is a geometric random variable. Let $X\sim \Geo{p=0.02}$.
$$P(X=10) ~=~ (1-0.02)^9(0.02) ~\approx~ \fbox{0.0167}$$
\item This is a special case of a binomial random variable.
$$P(\text{no defects in 100}) ~=~ 0.98^{100} ~\approx~ \fbox{0.133} $$
\item This is back to geometric.
$$\mu ~=~ \frac{1}{p} ~=~ \frac{1}{0.02} ~=~ 50$$
$$
\sigma ~=~ \frac{\sqrt{1-p}}{p} ~=~ \frac{\sqrt{1-0.02}}{0.02}~=~ 49.49747$$
On average, we expect to test 50 before we find a defective transistor, plus or minus 49.5.
\item We just change $p$.
$$\mu ~=~ \frac{1}{p} ~=~ \frac{1}{0.05} ~=~ 20$$

$$
\sigma ~=~ \frac{\sqrt{1-p}}{p} ~=~ \frac{\sqrt{1-0.05}}{0.05}~=~ 19.49359$$
\item Increasing $p$ decreases both $\mu$ and $\sigma$. Also, we might conjecture that when $p$ is small $\mu \approx \sigma + 0.5$. 
\end{enumerate}

\item \begin{enumerate}
\item We are dealing with a geometric distribution, where each trial has a $0.125$ chance of success. We let random variable $X$ represent the number of trials until success. 
$$X \sim \Geo{0.125} $$
We are asked to calculate $P(X=3)$.
$$P(X=3)~=~(1-0.125)^2(0.125) ~\approx~ 0.09570 $$
\item We are still considering a geometric distribution, so we use the appropriate formulas for (population) mean and (population) standard deviation. (Remember, distributions are infinitely large populations.)
$$\mu ~=~\frac{1}{p} ~=~ \frac{1}{0.125} ~=~ 8  $$ 

$$\sigma ~=~ \frac{\sqrt{1-p}}{p} ~=~ \frac{\sqrt{1-0.125}}{0.125}~\approx~ 7.483315$$
\end{enumerate}

\item We first need to remember normal distributions. Let normal random variable $X$ represent the speed of a car.
$$X\sim \N{72.6}{4.78} $$
We want to determine the probability that a single car is speeding.
$$P(X > 70) ~~=~~ 1- \Phi\left(\frac{70-72.6}{4.78}\right) ~~=~~ 0.7068 $$
I used precise software. If you are using the table, you can first estimate $z$.
$$z ~=~ \frac{70-72.6}{4.78} ~\approx~ -0.54 $$
Then, use the table...
$$P(X > 70) ~~\approx~~ \Phi(-0.54) ~~\approx~~ 0.7054  $$
We'll just go with that value; each car has a 70.5\% chance of speeding.
\begin{enumerate}
\item We will call speeding a ``success'' and not speeding a ``failure'' (even though speeding is bad). Thus, $p=0.705$. Also, this question asks for the probability of 5 failures.
$$P(\text{0 out of 5 cars speeding}) ~=~ (1-0.705)^5 ~=~ 0.0022  $$
\item Now we are considering a geometric distribution, so we use the appropriate formulas.
$$\mu ~=~ \frac{1}{0.705} ~\approx~ 1.42$$

$$\sigma ~=~ \frac{\sqrt{1-p}}{p} ~=~ \frac{\sqrt{1-0.705}}{0.705}~\approx~ 0.77$$
\end{enumerate}




\end{enumerate}
\end{document}
